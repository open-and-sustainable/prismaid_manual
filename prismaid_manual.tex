\documentclass{tufte-book}

% Font and Encoding
\usepackage{fontspec}
\setmainfont{Open Sans}[SmallCapsFont={}]

% Disable \nobibliography warning if no bibliography is used
\usepackage{etoolbox}
\AtBeginDocument{\let\nobibliography\relax}

% for the toc
\makeatletter
\renewcommand{\l@part}[2]{%
  \addvspace{1.5em}\noindent
  {\bfseries\Large\color{brightred}\MakeUppercase{Part #1}}\hfill
  {\bfseries\Large\color{brightred}#2}  % Correctly reference the page number
  \par\addvspace{1.0em}
}
\makeatother

\makeatletter
\renewcommand{\l@chapter}[2]{%
  \addvspace{1em}%
  {\bfseries\normalsize\color{brightred}#1}\hfill
  {\bfseries\normalsize\color{brightred}#2} % Ensures only the correct page number
  \par\addvspace{0.5em}
}
\makeatother

% for the parts titles
\usepackage{titlesec}  % Allows custom part title styling
\titleformat{\part}[display]
  {\Huge\bfseries\color{brightblue}}  % Ensures bright blue styling
  {\partname~\thepart}  % "Part X"
  {20pt}{}

% Updated Color Definitions (based on GIMP values)
\usepackage{xcolor}
\definecolor{brightred}{rgb}{0.976, 0.247, 0.118}  % Bright Red (Main Red)
\definecolor{darkred}{rgb}{0.631, 0.098, 0.047}    % Dark Red (Accent)
\definecolor{brightblue}{rgb}{0.0, 0.0, 1.0}       % Bright Blue (Main Blue)
\definecolor{darkblue}{rgb}{0.0, 0.0, 0.318}       % Dark Blue (Accent)

% Enable Page Numbers (Fix for tufte-book)
\usepackage{fancyhdr}
\fancypagestyle{plain}{%
    \fancyhf{}  % Clear header/footer
    \fancyfoot[C]{\thepage}  % Page number at the bottom center
    \renewcommand{\headrulewidth}{0pt}  % Remove header line
    \renewcommand{\footrulewidth}{0pt}  % Remove footer line
}
\pagestyle{plain}  % Apply numbering globally

% Colored boxes (less rounded)
\usepackage{tcolorbox}
\tcbuselibrary{skins}

% Colored boxes tyoes
\newtcolorbox{commandbox}[1][]{%
    colframe=brightblue, colback=brightblue!10, coltitle=white, % White title text
    boxrule=1pt, arc=1mm, left=5pt, right=5pt, top=5pt, bottom=5pt,
    fonttitle=\bfseries, title={#1}, colbacktitle=brightblue % Match background color
}

\newtcolorbox{configbox}[1][]{%
    colframe=brightred, colback=brightred!10, coltitle=white, % White title text
    boxrule=1pt, arc=1mm, left=5pt, right=5pt, top=5pt, bottom=5pt,
    fonttitle=\bfseries, title={#1}, colbacktitle=brightred % Match background color
}

% Code Formatting
\usepackage{listings}
\lstset{
    backgroundcolor=\color{darkblue!5},
    basicstyle=\ttfamily\footnotesize,
    frame=single,
    rulecolor=\color{darkblue},
    breaklines=true,
    keywordstyle=\color{brightblue}\bfseries,
    commentstyle=\color{darkred},
    stringstyle=\color{brightred}
}

% Define TOML Syntax
\lstdefinelanguage{TOML}{
    morekeywords={true,false},
    sensitive=true,
    morecomment=[l]{\#},  % TOML comments start with #
    morestring=[b]",      % Strings in double quotes
    morestring=[b]',      % Strings in single quotes
    keywordstyle=\color{brightblue}\bfseries,
    commentstyle=\color{darkred},
    stringstyle=\color{brightred},
    showspaces=false
}

% Define Bash Syntax
\lstdefinelanguage{Bash}{
    morekeywords={if, then, else, fi, for, while, do, done, exit, echo},
    sensitive=true,
    morecomment=[l]{\#},  % Comments start with #
    morestring=[b]",      % Double-quoted strings
    morestring=[b]',      % Single-quoted strings
    keywordstyle=\color{brightblue}\bfseries,
    commentstyle=\color{darkred},
    stringstyle=\color{brightred}
}

% Define Python Syntax
\lstdefinelanguage{Python}{
    morekeywords={def, return, if, elif, else, for, while, break, continue, import, from, as, class, try, except, with, yield, lambda, global, nonlocal, pass, raise, True, False, None},
    sensitive=true,
    morecomment=[l]{\#},  % Python comments start with #
    morestring=[b]",      % Double-quoted strings
    morestring=[b]',      % Single-quoted strings
    keywordstyle=\color{brightblue}\bfseries,
    commentstyle=\color{darkred},
    stringstyle=\color{brightred}
}

% Define Go Syntax
\lstdefinelanguage{Go}{
    morekeywords={package, import, func, return, var, const, type, struct, interface, map, range, if, else, for, switch, case, default, go, select, defer, break, continue, fallthrough},
    sensitive=true,
    morecomment=[l]{//},  % Single-line comment
    morecomment=[s]{/*}{*/}, % Multi-line comment
    morestring=[b]",      % Double-quoted strings
    morestring=[b]`,      % Backtick strings
    keywordstyle=\color{brightblue}\bfseries,
    commentstyle=\color{darkred},
    stringstyle=\color{brightred}
}

% Define R Syntax
\lstdefinelanguage{R}{
    morekeywords={if, else, repeat, while, function, for, in, next, break, TRUE, FALSE, NULL, NA, NaN, Inf},
    sensitive=true,
    morecomment=[l]{\#},  % R comments start with #
    morestring=[b]",      % Double-quoted strings
    morestring=[b]',      % Single-quoted strings
    keywordstyle=\color{brightblue}\bfseries,
    commentstyle=\color{darkred},
    stringstyle=\color{brightred}
}

% Define Julia Syntax
\lstdefinelanguage{Julia}{
    morekeywords={function, return, if, elseif, else, for, while, break, continue, import, using, struct, mutable, try, catch, finally, true, false, nothing},
    sensitive=true,
    morecomment=[l]{\#},  % Julia comments start with #
    morestring=[b]",      % Double-quoted strings
    morestring=[b]',      % Single-quoted strings
    keywordstyle=\color{brightblue}\bfseries,
    commentstyle=\color{darkred},
    stringstyle=\color{brightred}
}

% Colored Section Headings
\usepackage{sectsty}
\chapterfont{\color{brightred}}   % Chapter titles in bright red
\sectionfont{\color{darkred}}     % Section titles in dark red
\subsectionfont{\color{brightblue}} % Subsection titles in bright blue

% Hyperlinks
\usepackage{hyperref}
\hypersetup{
    colorlinks=true,
    linkcolor=brightred,
    urlcolor=brightblue,
    citecolor=darkred
}

% Define Custom Margin Note Commands
\newcommand{\tip}[1]{\marginnote{\textbf{\textcolor{brightblue}{Tip:}} #1}}
\newcommand{\warning}[1]{\marginnote{\textbf{\textcolor{brightred}{Warning:}} #1}}
\newcommand{\reminder}[1]{\marginnote{\textbf{\textcolor{darkred}{Reminder:}} #1}}
\newcommand{\note}[1]{\marginnote{\textbf{\textcolor{darkblue}{Note:}} #1}}

% Define a command to reference sections or chapters properly
\newcommand{\secref}[2]{\hyperref[#1]{\textbf{#2}}}

% for version tagging in document
\newcommand{\docversion}{\input{version.tex}}

% Define title manually to bypass \maketitle issues
\newcommand{\customtitle}{
    \begin{titlepage}
        \centering
        \vspace*{3cm}
        \begin{figure}
            \centering
            \includegraphics{figures/prismAId_logo.png}
        \end{figure}
        {\Huge\bfseries\sffamily prism\textcolor{brightred}{A}\textcolor{brightblue}{I}d User Manual}\\[2cm]
        {\Large\textit{Riccardo Boero}}\\[0.3cm]
        \texttt{ribo@nilu.no}\\[1cm]
        {\large\today}\\[0.3cm]  % Email
        {\large Version: \docversion}\\[2cm]  % tagged version
        DOI: \href{https://doi.org/10.5281/zenodo.15025694}{10.5281/zenodo.15025694}  % DOI link
        
        % Creative Commons License
        \vfill
        \noindent
        \begin{minipage}{0.8\textwidth}
            \centering
            \textbf{Licensed under CC BY-SA 4.0} \\[0.5cm]
            \includegraphics[width=3cm]{figures/by-sa.png} \\[0.5cm]  % CC logo (adjust path)
            This work is licensed under the 
            \href{https://creativecommons.org/licenses/by-sa/4.0/}{Creative Commons Attribution-ShareAlike 4.0 International License}.
        \end{minipage}
    \end{titlepage}
    \thispagestyle{plain}  % Force page number on the title page
}

\begin{document}
\lstset{showstringspaces=false}
\customtitle  % Use custom title instead of \maketitle

\tableofcontents  % Table of Contents after the title page
\newpage
\pagestyle{plain}  % Ensure numbering continues

\chapter*{Foreword}
\addcontentsline{toc}{chapter}{Foreword}
\markboth{FOREWORD}{FOREWORD}

The \texttt{prismAId} user manual is designed to help researchers, academics, and professionals leverage the power of the \texttt{prismAId} toolkit for conducting systematic reviews efficiently.\reminder{No coding skills are required to effectively conduct a systematic review with \texttt{prismAId}.} This document provides a structured approach to installing, configuring, and using the various components of \texttt{prismAId}, ensuring that users can quickly get started while also exploring advanced features when needed.

This manual is divided into five distinct parts, each catering to different user needs:

\begin{itemize}
    \item \textbf{Part 1: Introduction}\tip{The many open science advantages of \texttt{prismAId} are introduced and discussed in the \secref{chap:intro}{\textbf{Introduction}}.} -- If you are new to \texttt{prismAId}, we recommend starting with the chapter \secref{chap:intro}{\textbf{Introduction to prismAId}}. This section explains what \texttt{prismAId} is, the modular tools it provides, who can benefit from them, and why it is a valuable toolkit for systematic reviews.

    \item \textbf{Part 2: Getting Started}\note{\texttt{prismAId} is available on all platforms and operating systems. It can also be integrated programmatically with the most widely used scientific software.} -- If you need to install and configure \texttt{prismAId}, go directly to \secref{chap:install}{\textbf{Installation \& Setup}}. This section provides step-by-step installation instructions for Windows, Mac, and Linux.

    \item \textbf{Part 3: Conducting a Systematic Review}\tip{For a quick but complete walkthrough on using \texttt{prismAId} in a systematic review, see the \secref{chap:walkthrough}{\textbf{fifth}} chapter.} -- If your primary goal is to \textbf{learn how to conduct a systematic review with \texttt{prismAId}}, you can skip the installation and configuration sections and go directly to the chapter \secref{chap:walkthrough}{\textbf{Step-by-Step Guide to Conducting a Systematic Review}}. This section provides a hands-on walkthrough of:
    \begin{itemize}
        \item Setting up a project
        \item Acquiring and preparing literature using the Download and Convert tools
        \item Configuring and running systematic reviews using the Review tool
        \item Interpreting results
        \item Exporting findings
    \end{itemize}
    Before diving into this walkthrough, you may find it helpful to read \secref{chap:sysrev}{\textbf{Understanding Systematic Reviews}}, which provides background information on the methodology and best practices.

    \item \textbf{Part 4: Advanced Features} -- Users who want to explore advanced capabilities can refer to the chapters \secref{chap:advanced_ensemble}{\textbf{Ensemble Reviews}} and \secref{chap:advanced_features}{\textbf{Debugging, Cost Management \& Integration}}. These sections cover advanced options such as ensemble reviews, rate limiting, cost minimization, and integration with external tools and workflows.

    \item \textbf{Part 5: Troubleshooting \& FAQs} -- If you encounter issues, visit \secref{chap:troubleshooting}{\textbf{Troubleshooting Common Issues}}, which provides solutions to common errors for each of the toolkit's components, and \secref{chap:faq}{\textbf{Frequently Asked Questions}}, which addresses common concerns about usage and implementation.
\end{itemize}

\section*{How Should You Use This Manual?}
\addcontentsline{toc}{section}{How Should You Use This Manual?}

\begin{itemize}
    \item If you are \textbf{new to \texttt{prismAId}}, start with \secref{chap:intro}{\textbf{Introduction}}, then proceed to \secref{chap:install}{\textbf{Installation}}.
    \item If you \textbf{want to conduct a complete systematic review}, go directly to \textbf{Part 3}, particularly \secref{chap:walkthrough}{\textbf{Step-by-Step Guide to Conducting a Systematic Review}}, for a detailed walkthrough of the entire process.
    \item If you only need \textbf{specific tools} from the toolkit (Download, Convert, or Review), the respective chapters in \textbf{Part 3} detail how to use each component independently.
    \item If you need \textbf{advanced features or troubleshooting}, refer to \textbf{Parts 4 and 5} as needed.
\end{itemize}

Throughout this manual, useful commands are presented in blue boxes, as shown below:

\begin{commandbox}[Command: Example of a Binary Command]
To run a systematic review with a TOML configuration file:
\begin{lstlisting}[language=Bash]
# For Windows
./prismaid.exe -project your_project.toml

# For downloading papers from Zotero
./prismaid.exe -download-zotero zotero_config.toml

# For converting PDF files to text
./prismaid.exe -convert-pdf ./papers
\end{lstlisting}
\end{commandbox}

Tool configuration details are highlighted in red boxes, as shown below:

\begin{configbox}[Configuration: Example of a TOML Setting]
To configure an LLM for your review:
\begin{lstlisting}[language=TOML]
# Sample TOML configuration
[project.llm.1]
provider = "OpenAI"
api_key = ""
model = ""
temperature = 0.2
tpm_limit = 0
rpm_limit = 0
\end{lstlisting}
\end{configbox}


\section*{Final Notes}
\addcontentsline{toc}{section}{Final Notes}

We hope this manual serves as a comprehensive guide to making the most of the \texttt{prismAId} toolkit\warning{The online documentation of \texttt{prismAId} is available at \href{https://open-and-sustainable.github.io/prismaid/}{open-and-sustainable.github.io/prismaid/}.}. Whether you're a first-time user or an advanced researcher, this document is structured to help you get the information you need quickly and efficiently. The modular design of \texttt{prismAId} allows you to use only the components you need, making it adaptable to various research workflows and requirements.

This manual provides a complete overview of the features available in \texttt{prismAId} version ≥ 0.8.0. However, many aspects also apply to earlier versions that offer the same features.

For any additional questions, please refer to \secref{chap:faq}{the FAQ section} or contact our support team through the GitHub repository or Matrix Support Room. Happy reviewing!


% ==============================
% Part 1: Introduction
% ==============================
\part{Introduction}
\chapter{Introduction to prismAId} 
\label{chap:intro}

Systematic reviews are at the core of evidence-based research. They help synthesize vast amounts of literature, ensuring that decisions in science, medicine, policy, and other fields are grounded in the best available evidence. However, conducting systematic reviews is a time-intensive and often overwhelming process. Screening literature, managing citations, extracting data, and ensuring methodological rigor all demand meticulous attention to detail.

\texttt{prismAId} was created to address these challenges. It is an open-source tool designed to assist researchers in managing systematic reviews more efficiently.\tip{Systematic reviews require structured workflows. \texttt{prismAId} provides tools to help enforce best practices.} By streamlining key aspects of the review process, \texttt{prismAId} helps users maintain high standards of rigor and reproducibility while reducing the manual workload.

The tool is aimed at a diverse range of users, from researchers conducting large-scale systematic reviews to students working on literature-based projects. It provides structured workflows that align with best practices in systematic reviewing, ensuring that every step—from defining inclusion criteria to extracting and analyzing data—is traceable and transparent.

\bigskip

One of the key motivations behind \texttt{prismAId} is its commitment to Open Science. The tool is fully open source, meaning its development is transparent, and researchers can inspect, modify, and contribute to its functionality. This openness ensures that systematic reviews conducted with \texttt{prismAId} can be fully reproducible and that researchers can collaborate effectively without relying on proprietary or closed software ecosystems.\note{Open-source software allows full transparency. You can verify how \texttt{prismAId} processes data at any time.}

Moreover, the open-source nature of \texttt{prismAId} allows for continuous improvement through community-driven development. Users can adapt the tool to fit their needs, extend its capabilities, and integrate it with other research software.\warning{As an open-source tool, \texttt{prismAId} does not include proprietary support. Users are encouraged to engage with the community for troubleshooting and development.} This flexibility is particularly beneficial in a rapidly evolving scientific landscape where reproducibility and adaptability are crucial.

Beyond technical efficiency, \texttt{prismAId} embodies key guiding principles that impact its users:

\begin{itemize}
    \item \textbf{Transparency}: The logic behind how studies are managed and processed is open for review. There are no hidden decision-making mechanisms.
    \item \textbf{Reproducibility}: Every action taken within \texttt{prismAId} can be logged and traced, allowing others to replicate results.
    \item \textbf{Flexibility}: Researchers can configure \texttt{prismAId} to meet the specific requirements of their systematic review protocols.
    \item \textbf{No Vendor Lock-in}: Users are not dependent on a single commercial provider, ensuring that research workflows remain independent.
    \item \textbf{Community and Collaboration}:\reminder{Collaboration is a core feature. Consider contributing to \texttt{prismAId} if you have ideas for improvement!} As an open-source tool, \texttt{prismAId} fosters collaboration among researchers, developers, and systematic review specialists.
\end{itemize}


By embracing these principles, \texttt{prismAId} is more than just a tool—it is part of a broader effort to improve the accessibility, efficiency, and reliability of systematic reviews. Whether used by a lone researcher or a large team, it provides the structure and support needed to navigate complex literature and synthesize high-quality evidence.

As we move through this manual,\note{This manual follows a structured approach. Refer to the \textbf{How Should You Use This Manual?} section in the Foreword for guidance on navigating the content.} you will learn how to install, configure, and effectively use \texttt{prismAId} to conduct systematic reviews. From getting started with basic setup to leveraging advanced features for more complex reviews, this guide will provide all the necessary information to integrate \texttt{prismAId} into your research workflow.




% ==============================
% Part 2: Getting Started
% ==============================
\part{Getting Started}
\chapter{Installation \& Setup}
\label{chap:install}

This chapter explains how to install \texttt{prismAId}, covering system requirements, installation methods, and first-time setup for each of the toolkit's components.

\section{System Requirements}

Before installing \texttt{prismAId}, ensure your system meets the following requirements:\warning{Ensure that you have administrative privileges on your system before installing \texttt{prismAId}, especially on Windows and macOS.}

\begin{itemize}
    \item \textbf{Operating System}: Windows 10 or later, macOS 11 or later, or a Linux distribution (Arch, Debian, Fedora, Ubuntu, etc.).
    \item \textbf{Processor}: 64-bit CPU (Intel, AMD, or ARM64).
    \item \textbf{Memory}: At least 4GB RAM (8GB recommended).
    \item \textbf{Storage}: Minimum 500MB of free disk space.
    \item \textbf{Internet Connection}: Required for downloading packages and accessing LLM APIs.
    \item \textbf{API Keys}: To use the Review tool, you'll need an API key from at least one of the supported LLM providers (OpenAI, GoogleAI, Cohere, Anthropic, or DeepSeek).
\end{itemize}

\section{Installation Methods}

\texttt{prismAId} can be installed in multiple ways, depending on your preferences and workflow requirements.\warning{The capitalization of the letters in the prismAId tool follows the conventions of each programming language, ensuring consistency within each context but resulting in variations across packages and languages.}
 Choose the method that best suits your needs.

\subsection{Method 1: Standalone Binaries}

The simplest installation method is to download the pre-compiled binaries, which require no additional dependencies.\note{Binaries are standalone executables that work on Linux, macOS, and Windows, supporting both AMD64 and Arm64 architectures.}

\begin{enumerate}
    \item Navigate to the \href{https://github.com/open-and-sustainable/prismaid/releases}{GitHub Releases page}.
    \item Download the appropriate binary for your operating system.
    \item Place the file into a directory of your choice.
    \item On Linux and macOS, set the binaries as executable files.
\end{enumerate}

\subsubsection{Windows}
\reminder{On Windows, you may need to allow execution if prompted by security settings.}
\begin{commandbox}[Command: Running \texttt{prismAId} on Windows]
After downloading, navigate to the folder and run:
\begin{lstlisting}[language=Bash]
# Download papers from Zotero
./prismaid.exe -download-zotero zotero_config.toml

# Download papers from URL list
./prismaid.exe -download-URL paper_urls.txt

# Convert PDF files to text
./prismaid.exe -convert-pdf ./papers

# Run the Review tool with a project configuration
./prismaid.exe -project your_project.toml
\end{lstlisting}
\end{commandbox}


\subsubsection{macOS}
\warning{You may need to approve the application in \textbf{System Preferences} under \textbf{Security \& Privacy} before running it.}
\begin{commandbox}[Command: Running \texttt{prismAId} on macOS]
After downloading, give execution permissions and run:
\begin{lstlisting}[language=Bash]
chmod +x prismaid

# Download papers from Zotero
./prismaid -download-zotero zotero_config.toml

# Convert various file formats to text
./prismaid -convert-pdf ./papers
./prismaid -convert-docx ./papers
./prismaid -convert-html ./papers

# Run the Review tool with a project configuration
./prismaid -project your_project.toml
\end{lstlisting}
\end{commandbox}

\subsubsection{Linux}
\tip{If you receive a "Permission denied" error, try running \texttt{chmod +x} again or executing with \texttt{sudo}.}
\begin{commandbox}[Command: Running \texttt{prismAId} on Linux]
\begin{lstlisting}[language=Bash]
chmod +x prismaid

# Initialize a new project configuration interactively
./prismaid -init

# Download papers from a URL list
./prismaid -download-URL paper_urls.txt

# Run the Review tool with a project configuration
./prismaid -project your_project.toml
\end{lstlisting}
\end{commandbox}


\subsection{Method 2: Go Package}

If you are a Go developer, you can use \texttt{prismAId} as a Go package.\tip{The Go package offers the most comprehensive functionality, as it is the native implementation.}

\begin{commandbox}[Command: Installing the Go Package]
\begin{lstlisting}[language=Bash]
go get "github.com/open-and-sustainable/prismaid"
\end{lstlisting}
\end{commandbox}

\begin{commandbox}[Code: Using prismAId in Go]
\begin{lstlisting}[language=Go]
import "github.com/open-and-sustainable/prismaid"

// Download papers from Zotero
err := prismaid.DownloadZoteroPDFs(username, apiKey, collectionName, parentDir)

// Download from URL list
err := prismaid.DownloadURLList("path/to/urls.txt")

// Convert files to text
err := prismaid.Convert(inputDir, "pdf,docx,html")

// Run a systematic review
err := prismaid.Review(tomlConfigString)
\end{lstlisting}
\end{commandbox}

\subsection{Method 3: Python Package}

For Python users, \texttt{prismAId} is available as a package on PyPI.\note{The Python package works on Linux and Windows AMD64, and macOS Arm64.}

\begin{commandbox}[Command: Installing the Python Package]
\begin{lstlisting}[language=Bash]
pip install prismaid
\end{lstlisting}
\end{commandbox}

\begin{commandbox}[Code: Using prismAId in Python]
\begin{lstlisting}[language=Python]
import prismaid

# Download papers from Zotero
prismaid.download_zotero_pdfs("username", "api_key", "collection_name", "./papers")

# Download from URL list
prismaid.download_url_list("urls.txt")

# Convert files to text
prismaid.convert("./papers", "pdf,docx,html")

# Run a systematic review
with open("project.toml", "r") as file:
    toml_config = file.read()
prismaid.review(toml_config)
\end{lstlisting}
\end{commandbox}

\subsection{Method 4: R Package}

For R users, \texttt{prismAId} is available as a package on R-universe.\note{The R package works on Linux AMD64 and macOS Arm64.}

\begin{commandbox}[Command: Installing the R Package]
\begin{lstlisting}[language=R]
install.packages("prismaid", repos = c("https://open-and-sustainable.r-universe.dev", "https://cloud.r-project.org"))
\end{lstlisting}
\end{commandbox}

\begin{commandbox}[Code: Using prismAId in R]
\begin{lstlisting}[language=R]
library(prismaid)

# Download papers from Zotero
DownloadZoteroPDFs("username", "api_key", "collection_name", "./papers")

# Download from URL list
DownloadURLList("urls.txt")

# Convert files to text
Convert("./papers", "pdf,docx,html")

# Run a systematic review
toml_content <- paste(readLines("project.toml"), collapse = "\n")
RunReview(toml_content)  # Note the capitalization
\end{lstlisting}
\end{commandbox}

\subsection{Method 5: Julia Package}

For Julia users, \texttt{prismAId} is available as a package published in the Julia General Registry.\note{The Julia package works on Linux and Windows AMD64, and macOS Arm64.}

\begin{commandbox}[Command: Installing the Julia Package]
\begin{lstlisting}[language=Julia]
using Pkg
Pkg.add("PrismAId")
\end{lstlisting}
\end{commandbox}

\begin{commandbox}[Code: Using prismAId in Julia]
\begin{lstlisting}[language=Julia]
using PrismAId

# Download papers from Zotero
PrismAId.download_zotero_pdfs("username", "api_key", "collection_name", "./papers")

# Download from URL list
PrismAId.download_url_list("urls.txt")

# Convert files to text
PrismAId.convert("./papers", "pdf,docx,html")

# Run a systematic review
toml_config = read("project.toml", String)
PrismAId.run_review(toml_config)
\end{lstlisting}
\end{commandbox}

\section{Setting Up API Keys}

To use the Review tool of \texttt{prismAId}, you need to set up API keys for at least one of the supported LLM providers.\warning{Keep your API keys secure. Never share them in public repositories or unencrypted communications.}

\subsection{Obtaining API Keys}

\begin{itemize}
    \item \textbf{Anthropic}: Register at \href{https://www.anthropic.com/}{anthropic.com} and generate an API key.
    \item \textbf{Cohere}: Sign up at \href{https://cohere.com/}{cohere.com} and retrieve your API key from the dashboard.
    \item \textbf{DeepSeek}: Create an account at \href{https://platform.deepseek.com/}{platform.deepseek.com} and obtain an API key.
    \item \textbf{GoogleAI}: Create an account at \href{https://aistudio.google.com}{aistudio.google.com} and generate an API key.
    \item \textbf{OpenAI}: Register at \href{https://www.openai.com/}{openai.com} and obtain an API key from your account dashboard.
\end{itemize}

\subsection{Configuring API Keys}

There are two ways to configure your API keys:

\subsubsection{Environment Variables}

Set environment variables for your API keys:

\begin{commandbox}[Command: Setting API Key Environment Variables]
\begin{lstlisting}[language=Bash]
# For OpenAI
export OPENAI_API_KEY="your-openai-api-key"

# For GoogleAI
export GOOGLEAI_API_KEY="your-googleai-api-key"

# For Cohere
export COHERE_API_KEY="your-cohere-api-key"

# For Anthropic
export ANTHROPIC_API_KEY="your-anthropic-api-key"

# For DeepSeek
export DEEPSEEK_API_KEY="your-deepseek-api-key"
\end{lstlisting}
\end{commandbox}

\subsubsection{Configuration File}

Add your API keys directly in the TOML configuration file:\warning{Do not share or publish configuration files containing API keys. Always remove them before sharing.}

\begin{configbox}[Configuration: API Keys in TOML]
\begin{lstlisting}[language=TOML]
[project.llm.1]
provider = "OpenAI"
api_key = "your-openai-api-key"
model = "gpt-4o-mini"
temperature = 0.01
tpm_limit = 0
rpm_limit = 0
\end{lstlisting}
\end{configbox}

\tip{If both environment variables and configuration file entries are present, the configuration file values take priority.}

\section{Verifying the Installation}

Check if \texttt{prismAId} is correctly installed:\note{If you encounter issues, refer to the troubleshooting section or seek help from the community through GitHub issues or the Matrix Support Room.}

\begin{commandbox}[Command: Verifying Installation]
\begin{lstlisting}[language=Bash]
# For binaries
./prismaid --help

# For Python
python -c "import prismaid; print(prismaid.__version__)"

# For R
R -e "library(prismaid); cat(prismaid_version())"

# For Julia
julia -e "using PrismAId; println(PrismAId.version())"
\end{lstlisting}
\end{commandbox}

\section{Use in Jupyter Notebooks}

When using prismAId (versions ≤ 0.6.6) in Jupyter notebooks with Python, special handling may be required for interactive prompts:

\begin{commandbox}[Code: Using prismAId v < 0.6.6 in Jupyter Notebooks]
\begin{lstlisting}[language=Python]
import pty
import os
import time
import select

def run_review_with_auto_input(input_str):
    master, slave = pty.openpty()  # Create a pseudo-terminal

    pid = os.fork()
    if pid == 0:  # Child process
        os.dup2(slave, 0)  # Redirect stdin
        os.dup2(slave, 1)  # Redirect stdout
        os.dup2(slave, 2)  # Redirect stderr
        os.close(master)
        import prismaid
        prismaid.RunReviewPython(input_str.encode("utf-8"))
        os._exit(0)

    else:  # Parent process
        os.close(slave)
        try:
            while True:
                rlist, _, _ = select.select([master], [], [], 5)
                if master in rlist:
                    output = os.read(master, 1024).decode("utf-8", errors="ignore")
                    if not output:
                        break  # Process finished

                    print(output, end="")

                    if "Do you want to continue?" in output:
                        print("\n[SENDING INPUT: y]")
                        os.write(master, b"y\n")
                        time.sleep(1)
        finally:
            os.close(master)
            os.waitpid(pid, 0)  # Ensure the child process is cleaned up

# Load your review (TOML) configuration
with open("config.toml", "r") as file:
    input_str = file.read()

# Run the review function
run_review_with_auto_input(input_str)
\end{lstlisting}
\end{commandbox}

\section{Conclusion}

You have now installed \texttt{prismAId} and completed the first-time setup. The toolkit offers five different installation methods, making it accessible across various platforms and programming environments. Each component (Download, Convert, and Review) can be used independently or as part of an integrated workflow.

In the next chapter, we'll explore how to configure your project in detail to leverage all of \texttt{prismAId}'s capabilities.


% ==============================
% Part 3: Conducting a Systematic Review
% ==============================
\part{Conducting a Systematic Review}
\chapter{Understanding Systematic Reviews} \label{chap:sysrev}

This chapter provides a foundational understanding of systematic reviews, their methodological framework, and best practices that \texttt{prismAId} is designed to support.

\section{What is a Systematic Review?}

A systematic review is a rigorous, transparent approach to synthesizing existing research literature on a specific research question.\tip{The term "systematic" emphasizes that these reviews follow an explicit, reproducible methodology.} Unlike narrative reviews, which offer subjective summaries of selected studies, systematic reviews aim to identify, evaluate, and integrate findings from all relevant studies using predefined protocols.

Systematic reviews serve several critical purposes in scientific research:\note{Systematic reviews differ from other literature reviews in their methodological rigor, comprehensive search strategies, explicit inclusion criteria, and transparent reporting of methods.}

\begin{itemize}
    \item \textbf{Consolidating Knowledge}: They integrate findings across multiple studies, providing a comprehensive understanding of available evidence.
    \item \textbf{Identifying Gaps}: By mapping existing literature, they reveal knowledge gaps and research opportunities.
    \item \textbf{Minimizing Bias}: Their structured approach reduces subjective biases in literature selection and interpretation.
    \item \textbf{Supporting Evidence-Based Decisions}: They provide synthesized evidence to inform policy, practice, and further research.
\end{itemize}

Systematic reviews can also include meta-analyses, which statistically combine results from multiple studies to estimate overall effects.\warning{Meta-analyses require statistical expertise and should only be conducted when studies are sufficiently similar in design and outcome measures.}

\section{Key Concepts \& Methodology}

The systematic review process follows a well-established methodology that ensures transparency and reproducibility. The widely adopted PRISMA framework (Preferred Reporting Items for Systematic Reviews and Meta-Analyses) outlines the following essential steps:

\begin{infobox}[PRISMA 2020 Framework Overview]
\begin{lstlisting}
PRISMA 2020 Key Components:
1. Title & Abstract
2. Introduction (Rationale, Objectives)
3. Methods (Eligibility criteria, Information sources, Search strategy,
   Selection process, Data extraction, Study quality assessment)
4. Results (Study selection, Study characteristics, Risk of bias,
   Results of syntheses, Reporting biases)
5. Discussion (Summary, Limitations, Conclusions)
6. Other information (Registration, Protocol, Support)

Full checklist available at: prisma-statement.org
\end{lstlisting}
\end{infobox}

\subsection{Pre-Review Planning}

Before beginning the review, researchers must:\tip{Well-formulated research questions are specific, clear, and focused. Avoid questions that are too broad ("What is known about climate change?") or too narrow ("What is the effect of a 1.5°C temperature increase on the reproduction of a specific butterfly species in northern Sweden?").}

\begin{itemize}
    \item \textbf{Formulate Research Questions}: Define specific, answerable questions using frameworks like PICO (Population, Intervention, Comparison, Outcome) or PEO (Population, Exposure, Outcome).
    \item \textbf{Develop Review Protocol}: Create a detailed plan specifying search strategies, inclusion/exclusion criteria, and analysis methods.\reminder{Always register your systematic review protocol before beginning the review process. This prevents bias and demonstrates methodological transparency.}

    \item \textbf{Register Protocol}: Pre-register the protocol in repositories like PROSPERO to enhance transparency and prevent duplication.
\end{itemize}

\begin{infobox}[PICO Framework Example]
\begin{lstlisting}
Research Question: "What is the effect of cognitive behavioral therapy compared to medication on depression symptoms in adults?"

P (Population): Adults with depression
I (Intervention): Cognitive behavioral therapy
C (Comparison): Medication treatment
O (Outcome): Depression symptom reduction
\end{lstlisting}
\end{infobox}

\subsection{Literature Search \& Selection}

The search and selection phase involves:

\begin{itemize}
    \item \textbf{Comprehensive Search}: Implement systematic search strategies across multiple databases using carefully constructed search terms.\warning{Relying on a single database significantly increases the risk of missing relevant studies. Research shows that even comprehensive databases like PubMed or Web of Science individually capture only 50-75\% of eligible studies in many fields.}
    \item \textbf{Screening}: Apply inclusion and exclusion criteria in a two-stage process: title/abstract screening followed by full-text assessment.\note{prismAId's Screening tool can automate the initial filtering of manuscripts using deduplication, language detection, article type classification, and topic relevance scoring, significantly reducing the manual workload before full-text download.}
    \item \textbf{PRISMA Flow Diagram}: Document the selection process, including numbers of studies identified, screened, assessed for eligibility, and included.
\end{itemize}

\begin{infobox}[PubMed Search Strategy Example]
\begin{lstlisting}
("climate change"[MeSH Terms] OR "global warming"[Title/Abstract])
AND
("agriculture"[MeSH Terms] OR "crop yield"[Title/Abstract] OR "food production"[Title/Abstract])
AND
("adaptation"[Title/Abstract] OR "mitigation"[Title/Abstract])
AND
("2010"[Date - Publication] : "2023"[Date - Publication])
\end{lstlisting}
\end{infobox}

\subsection{Data Extraction \& Quality Assessment}

Once relevant studies are identified, researchers:

\begin{itemize}
    \item \textbf{Extract Data}: Systematically collect relevant information from each study using standardized forms.\tip{Create your data extraction form in digital format (e.g., spreadsheet) rather than paper. This facilitates easier data manipulation, sharing among team members, and integration with analysis software.}
    \item \textbf{Assess Quality}: If a manual review, evaluate methodological rigor using tools like the Cochrane Risk of Bias Tool or GRADE (Grading of Recommendations Assessment, Development, and Evaluation). If AI-assisted, follow approaches described in this manual.
    \item \textbf{Document Uncertainties}: Record ambiguities or missing information, often contacting original authors for clarification.
\end{itemize}

\begin{infobox}[Data Extraction Template Example]
\begin{lstlisting}
Study ID: [Reference ID]
Citation: [Full citation in standard format]
Study Design: [RCT, cohort, case-control, etc.]
Population Characteristics:
  - Sample size: [n=]
  - Demographics: [age, gender, location, etc.]
  - Inclusion criteria: [as reported]
Intervention/Exposure:
  - Type: [specific details]
  - Duration: [time period]
  - Frequency: [how often applied]
Comparison/Control: [details of control group]
Outcomes:
  - Primary: [specific measure, unit]
  - Secondary: [additional measures]
Results:
  - Main findings: [effect sizes, p-values]
  - Subgroup analyses: [if applicable]
Study Quality Assessment:
  - Tool used: [e.g., Cochrane RoB, GRADE]
  - Overall rating: [Low/Moderate/High risk of bias]
  - Key limitations: [specific methodological concerns]
Notes: [Additional relevant information]
\end{lstlisting}
\end{infobox}

\subsection{Synthesis \& Analysis}

The final analytical phase includes:

\begin{itemize}
    \item \textbf{Narrative Synthesis}: Organize and summarize findings qualitatively, identifying patterns and relationships.
    \item \textbf{Quantitative Synthesis}: When appropriate, conduct meta-analysis to statistically combine results.\note{Not all systematic reviews are suitable for meta-analysis. When studies use different outcome measures or have high methodological heterogeneity, narrative synthesis may be more appropriate.}
    \item \textbf{Heterogeneity Assessment}: Evaluate variations in study designs, populations, and outcomes.
    \item \textbf{Subgroup Analysis}: Explore how findings vary across different study characteristics or populations.
\end{itemize}
\warning{Systematic reviews require significant time and resources. A comprehensive review typically takes 6-18 months to complete when conducted manually. Plan your timeline accordingly and consider the efficiency benefits tools like \texttt{prismAId} can provide.}
\begin{commandbox}[R Code Example for Forest Plot Creation]
\begin{lstlisting}[language=R]
# Basic R code for meta-analysis and forest plot
library(meta)

# Create meta-analysis object
meta_analysis <- metagen(TE = effect_size,
                         seTE = standard_error,
                         studlab = study_name,
                         data = extracted_data,
                         sm = "SMD")

# Generate forest plot
forest(meta_analysis,
       sortvar = year,
       prediction = TRUE,
       print.tau2 = TRUE,
       print.I2 = TRUE)
\end{lstlisting}
\end{commandbox}

\section{Best Practices}

Conducting high-quality systematic reviews involves adhering to several best practices:

\subsection{Transparency \& Reproducibility}

\begin{itemize}
    \item \textbf{Detailed Documentation}: Record all decisions, search strategies, and methods.\reminder{Document all exclusion decisions during screening. For transparency, maintain a list of studies excluded at the full-text review stage with specific reasons for exclusion.}
    \item \textbf{PRISMA Compliance}: Follow PRISMA guidelines for reporting.
    \item \textbf{Open Data}: When possible, share data extraction forms and analysis files.
\end{itemize}

\subsection{Methodological Rigor}

\begin{itemize}
    \item \textbf{Comprehensive Searching}: Use multiple databases and supplementary methods like citation tracking.
    \item \textbf{Dual Screening}: If a manual review, have at least two independent reviewers screen studies, with a third resolving disagreements.\tip{Again if you do not want to take advantage of prismAId and you are proceeding with a manual review, before full screening, conduct a calibration exercise where all reviewers independently screen the same 5-10 studies, then compare results to ensure consistent application of criteria. This identifies misunderstandings early and improves inter-rater reliability.}
    \item \textbf{Pilot Testing}: Test screening and data extraction procedures on a subset of studies.
\end{itemize}

\begin{infobox}[Screening Criteria Documentation]
\begin{lstlisting}
Inclusion Criteria:
- Population: Adults (18+) with Type 2 Diabetes
- Intervention: Digital health interventions
- Comparison: Standard care or other interventions
- Outcomes: HbA1c levels, quality of life measures
- Study Design: Randomized controlled trials
- Publication: Peer-reviewed, English language, 2010-2023

Exclusion Criteria:
- Studies focusing exclusively on Type 1 Diabetes
- Non-digital interventions
- Conference abstracts or proceedings
- Studies without control groups
\end{lstlisting}
\end{infobox}

\subsection{Managing Bias}

\begin{itemize}
    \item \textbf{Publication Bias}: Search for unpublished studies and conduct funnel plot analysis when applicable.\warning{Be cautious of language bias. Limiting searches to English-language publications can miss important evidence, particularly in fields with significant international research or regional focus.}

    \item \textbf{Selection Bias}: Use predefined, objective inclusion criteria.
    \item \textbf{Confirmation Bias}: If a manual review, have team members with diverse perspectives review the evidence.
\end{itemize}

\begin{commandbox}[Code for Publication Bias Assessment]
\begin{lstlisting}[language=R]
# Funnel plot and Egger's test in R
library(meta)

# Create funnel plot
funnel(meta_analysis,
       studlab = TRUE,
       contour = TRUE)

# Perform Egger's test for publication bias
metabias(meta_analysis, method = "linreg")
\end{lstlisting}
\end{commandbox}

\subsection{Team Collaboration}

\begin{itemize}
    \item \textbf{Multidisciplinary Teams}: Ideally, include subject matter experts, methodologists, and librarians.\note{Consulting with a research librarian can significantly improve search strategy quality. Their expertise in database syntax and controlled vocabulary (e.g., MeSH terms) helps ensure comprehensive literature identification.}\reminder{The most robust manual systematic reviews often involve teams rather than individual researchers. If working alone, consider consulting with colleagues at critical decision points to reduce subjectivity and to take advantage of prismAId.}
    \item \textbf{Regular Calibration}: Conduct ongoing discussions to ensure consistent application of criteria.
    \item \textbf{Clear Roles}: If there is a team, define responsibilities for each team member.
\end{itemize}

\section{The Role of Technology in Systematic Reviews}

Traditional systematic review methods are resource-intensive and time-consuming. Recent technological advances have created opportunities to streamline the process while maintaining methodological rigor:

\begin{itemize}
    \item \textbf{Literature Screening}: Machine learning algorithms can help prioritize relevant studies. \texttt{prismAId}'s Screening tool automates this critical step by applying deduplication, language detection, article type classification, and topic relevance scoring to filter manuscripts before full-text download.\tip{Even with technological assistance, allocate time for pilot testing and validation. Test \texttt{prismAId} with a small subset of your literature before processing your entire dataset.}
    \item \textbf{Data Extraction}: Natural language processing can identify key information from texts. This is what the prismAId Review function does.
    \item \textbf{Evidence Synthesis}: Automated tools can assist in organizing and analyzing extracted data.
    \item \textbf{Protocol Management}: Specialized software can guide teams through the systematic review workflow. Other prismAId functions like Download and Convert support specific steps of the workflow.
\end{itemize}

\texttt{prismAId} contributes to the evolution of this technological landscape, leveraging advanced large language models to assist with data extraction in the systematic review process.

\subsection{Benefits of Technology-Assisted Reviews}

Technology-assisted systematic reviews offer several advantages:\warning{While AI tools like \texttt{prismAId} significantly enhance efficiency, human oversight remains essential. AI systems may miss nuanced information or misinterpret specialized terminology. Always validate AI-extracted data against source documents.}

\begin{itemize}
    \item \textbf{Increased Efficiency}: Reducing time required for screening and data extraction.
    \item \textbf{Enhanced Consistency}: Applying criteria uniformly across all studies.
    \item \textbf{Improved Scalability}: Managing larger volumes of literature.
    \item \textbf{Better Reproducibility}: Creating structured, traceable processes.
\end{itemize}

\section{Conclusion}

\tip{Start small and scale up. If you're new to systematic reviews, consider conducting a scoping review or rapid review on a narrow topic before undertaking a comprehensive systematic review.}Systematic reviews are cornerstone methods for evidence synthesis across scientific disciplines. By following rigorous methodologies and best practices, researchers can produce high-quality reviews that reliably inform decision-making and future research directions.


\reminder{Even the most sophisticated tools can't replace critical thinking. Use \texttt{prismAId} to handle repetitive tasks, but apply your subject expertise to interpret findings in context and derive meaningful conclusions.}\texttt{prismAId} is designed to support this methodology by automating labor-intensive, error-prone, and intrinsically subjective aspects of the review process while maintaining and enhancing the methodological rigor that makes systematic reviews valuable. In the following chapters, we'll explore how to harness \texttt{prismAId}'s capabilities to conduct efficient, protocol-driven systematic reviews.


\chapter[Step-by-Step to a Systematic Review]{Step-by-Step Guide to Conducting a Systematic Review} \label{chap:walkthrough}

This chapter provides a comprehensive, practical walkthrough of conducting a systematic review using \texttt{prismAId}, from initial setup to final export of findings. Follow these steps sequentially to complete your review efficiently.

\section{Setting Up a Project}

Before diving into the technical aspects, proper project planning is essential for a successful systematic review.

\subsection{Sketch Your Research Question}

\begin{enumerate}
    \item Formulate a clear, focused research question using a framework like PICO.\tip{Well-defined research questions lead to more precise information extraction. Be specific about what you want to learn from the literature.}
    \item Outline the scope of your review (time period, study types, etc.).
    \item Determine what specific information you need to extract from each paper.\note{Iterative piloting, learning, narrowing the focus, and simplifying research questions, search queries, and methodological design are crucial for achieving higher quality and replicability standards.}
\end{enumerate}

To support better scoping of your future work, it is helpful to fill in and revise a review registration. Beyond being an important step in the review process, using a registration template helps focus on the right questions and identify issues that need clarification.

\subsection{Create a Project Directory Structure}
\reminder{A well-organized directory structure makes it easier to track your workflow and prevents confusion between original documents and processed files.}

\begin{commandbox}[Command: Creating a Project Directory Structure]
\begin{lstlisting}[language=Bash]
# Create main project directory
mkdir my_systematic_review

# Create subdirectories for different stages
mkdir my_systematic_review/papers_pdf
mkdir my_systematic_review/papers_txt
mkdir my_systematic_review/config
mkdir my_systematic_review/results
\end{lstlisting}
\end{commandbox}

\section{Screening Manuscripts}

After conducting your literature search and before downloading full texts, use the Screening tool to filter manuscripts efficiently and reduce the volume of papers requiring full review.

\begin{figure}[h]
    \centering
    \includegraphics[width=\textwidth]{figures/screening_tools.png}
    \caption{The prismAId Screening tool applies multiple filters in sequence: deduplication, language detection, article type classification, and topic relevance scoring. Each filter can be configured independently to match your systematic review protocol.}
    \label{fig:screening_tools}
\end{figure}

\subsection{Why Screen Before Downloading?}

\begin{itemize}
    \item \textbf{Resource Efficiency}: Avoid downloading and processing irrelevant papers.
    \item \textbf{Time Savings}: Focus manual review efforts on truly relevant literature.
    \item \textbf{Systematic Approach}: Apply consistent inclusion/exclusion criteria across all manuscripts.
    \item \textbf{Audit Trail}: Document exclusion decisions for transparency and reproducibility.
\end{itemize}

\subsection{Preparing Your Screening Configuration}

\begin{configbox}[Configuration: Screening Setup]
\begin{lstlisting}[language=TOML]
# screening_config.toml
[project]
name = "Climate Change Agriculture Screening"
author = "Your Name"
version = "1.0"
input_file = "search_results.csv"  # Export from your database search
output_file = "screening_results"
text_column = "abstract"           # Column containing abstracts
identifier_column = "doi"          # Unique identifier column
output_format = "csv"
log_level = "medium"

[filters.deduplication]
enabled = true
compare_fields = ["title", "doi"]

[filters.language]
enabled = true
accepted_languages = ["en"]

[filters.article_type]
enabled = true
exclude_reviews = true
exclude_editorials = true
exclude_letters = true

[filters.topic_relevance]
enabled = true
topics = ["climate change AND agriculture",
          "crop yield AND temperature"]
min_score = 0.5
\end{lstlisting}
\end{configbox}

\subsection{Running the Screening Tool}

\begin{commandbox}[Command: Screening Manuscripts]
\begin{lstlisting}[language=Bash]
# Run screening on your search results
./prismaid --screening config/screening_config.toml

# Review the screening results
cat results/screening_results.csv | head -n 10
\end{lstlisting}
\end{commandbox}

\tip{Start with conservative filtering settings. You can always apply stricter criteria in a second pass, but you cannot recover manuscripts that were incorrectly excluded.}

\warning{Always manually review a sample of excluded manuscripts to ensure your screening criteria are not too restrictive. Check both included and excluded items for false positives and false negatives.}

\subsection{Understanding Screening Results}

The Screening tool outputs a CSV or JSON file containing:
\begin{itemize}
    \item Original manuscript metadata
    \item Inclusion/exclusion decision
    \item Exclusion reasons (if applicable)
    \item Filter-specific tags (detected language, article type, relevance score)
\end{itemize}

\note{Manuscripts passing all filters will have \texttt{include = true} and can proceed to the download phase. Those excluded will have clear reasons documented for your PRISMA flow diagram.}

\section{Importing and Managing Manuscripts}

Once your project and search are set up, and you've screened your initial results, you'll need to gather and prepare your literature for analysis.

\subsection{Literature Acquisition using the Download Tool}

\subsubsection{Option 1: Downloading from Zotero}\warning{Manuscripts are often protected by copyright. Make sure to respect all applicable rights and use them only within permitted conditions.}
\note{The collection path in Zotero follows a filesystem-like format. For example, "Parent Collection/Sub Collection" or "Group Name/Collection Name" for group libraries.}
\begin{enumerate}
    \item Prepare your Zotero collection with all papers for your systematic review.
    \item Obtain your Zotero user ID and API key.
    \item Create a Zotero configuration file or use command line options.
\end{enumerate}

\begin{configbox}[Configuration: Zotero Download Setup]
\begin{lstlisting}[language=TOML]
# zotero_config.toml
user = "123456789"  # Your Zotero user ID
api_key = "AbCdEfGhIjKlMnOpQrStUv"  # Your Zotero API key
group = "Climate Change Research/Agriculture Papers"  # Your collection path
\end{lstlisting}
\end{configbox}

\begin{commandbox}[Command: Downloading Papers from Zotero]
\begin{lstlisting}[language=Bash]
# Navigate to your project directory
cd my_systematic_review

# Download papers to the papers_pdf directory
./prismaid -download-zotero config/zotero_config.toml -o papers_pdf
\end{lstlisting}
\end{commandbox}

\subsubsection{Option 2: Downloading from URL Lists}
\warning{Some publishers may restrict automatic downloads. Ensure you have appropriate access rights to all papers before attempting to download them.}
\begin{enumerate}
    \item Create a text file with one URL per line for papers you want to download.
    \item Run the URL download command to fetch all papers.
\end{enumerate}

\begin{infobox}[Example URL List File: paper\_urls.txt]
\begin{lstlisting}
https://arxiv.org/pdf/2303.08774.pdf
https://www.science.org/doi/pdf/10.1126/science.1236498
https://www.nature.com/articles/s41586-021-03819-2.pdf
\end{lstlisting}
\end{infobox}

\begin{commandbox}[Command: Downloading Papers from URL List]
\begin{lstlisting}[language=Bash]
# Navigate to your project directory
cd my_systematic_review

# Download papers to the papers_pdf directory
./prismaid -download-URL config/paper_urls.txt -o papers_pdf
\end{lstlisting}
\end{commandbox}



\subsection{Document Conversion using the Convert Tool}

After downloading your papers, you need to convert them to plain text format for analysis.

\begin{commandbox}[Command: Converting PDF Files to Text]
\begin{lstlisting}[language=Bash]
# Navigate to your project directory
cd my_systematic_review

# Convert all PDFs in papers_pdf directory to text files in papers_txt
./prismaid -convert-pdf papers_pdf -o papers_txt
\end{lstlisting}
\end{commandbox}

\reminder{Always manually check a sample of converted files to ensure the conversion quality is acceptable. PDFs with complex layouts or scanned images may not convert perfectly.}
\begin{commandbox}[Command: Converting Multiple File Types]
\begin{lstlisting}[language=Bash]
# For DOCX files
./prismaid -convert-docx papers_docx -o papers_txt

# For HTML files
./prismaid -convert-html papers_html -o papers_txt
\end{lstlisting}
\end{commandbox}

\subsection{Organizing Your Literature Collection}

\begin{enumerate}
    \item Review the converted text files for quality and completeness.\tip{Consider removing unnecessary sections from converted texts, such as references or acknowledgments, to reduce token usage and improve extraction accuracy.}
    \item Remove or fix any problematic conversions.
    \item Optionally, rename files for better organization and tracking.
\end{enumerate}


\subsection{Configure Your Project}

\begin{enumerate}
    \item Generate a basic configuration file using the \texttt{prismAId} initializer.
    \item Customize the configuration file to match your research questions.
\end{enumerate}

You can follow templates and example or create a new project configuration interactively or by using the web-based configurator.

\subsubsection{Interactive Terminal Setup}
\reminder{Always keep a backup of your configuration file before making major modifications.}
\begin{commandbox}[Command: Creating a New Project]
Run the following command:
\begin{lstlisting}[language=Bash]
./prismaid -init
\end{lstlisting}
\end{commandbox}

This prompts you with multiple questions and generates a TOML configuration file.

\subsubsection{Web-Based Setup}

Alternatively, use the web-based configurator available at \href{https://open-and-sustainable.github.io/prismaid/review-configurator.html}{open-and-sustainable.github.io/prismaid/review-configurator.html} to create your configuration file interactively in a browser.

\begin{configbox}[Configuration: Example of Initial Sections of a Configuration]
To configure \texttt{prismAId}, edit the generated \texttt{your\_project.toml} file:
\begin{lstlisting}[language=TOML]
[project]
name = "Use of LLM for systematic review"
author = "John Doe"
version = "1.0"

[project.configuration]
input_directory = "/path/to/txt/files"
results_file_name = "/path/to/save/results"
output_format = "json"
log_level = "low"
duplication = "no"
cot_justification = "no"
summary = "no"
\end{lstlisting}
\end{configbox}
\note{Alternatively, you can use the web-based configurator at \href{https://open-and-sustainable.github.io/prismaid/review-configurator.html}{open-and-sustainable.github.io/prismaid/review-configurator.html} to create your configuration file with a user-friendly interface.}
\begin{commandbox}[Command: Initializing a Project Configuration]
\begin{lstlisting}[language=Bash]
# Navigate to your project directory
cd my_systematic_review

# Initialize a new configuration file
./prismaid -init
\end{lstlisting}
\end{commandbox}

\warning{Always use absolute paths in your configuration file to prevent path-related errors. Relative paths can cause issues when running \texttt{prismAId} from different directories.}
\begin{configbox}[Configuration: Essential Project Settings]
\begin{lstlisting}[language=TOML]
[project]
name = "Effects of Climate Change on Agricultural Yields"
author = "Jane Researcher"
version = "1.0"

[project.configuration]
input_directory = "/home/user/my_systematic_review/papers_txt"
results_file_name = "/home/user/my_systematic_review/results/findings"
output_format = "csv"
log_level = "medium"
duplication = "no"
cot_justification = "yes"
summary = "yes"
\end{lstlisting}
\end{configbox}


\subsection{Configure LLM Settings}

\begin{enumerate}
    \item Obtain API key(s) from your preferred LLM provider(s).\tip{Using a lower temperature setting (0.01-0.1) produces more consistent results, which is generally preferable for systematic reviews where reproducibility is important.}
    \item Set up environment variables or add keys to your configuration file.
    \item Define model settings based on your needs and budget.
\end{enumerate}

\begin{configbox}[Configuration: LLM Provider Setup]
\begin{lstlisting}[language=TOML]
[project.llm.1]
provider = "OpenAI"
api_key = "" # Leave empty to use environment variable
model = "gpt-4o-mini"
temperature = 0.01
tpm_limit = 0
rpm_limit = 0
\end{lstlisting}
\end{configbox}

\section{Running Analyses}

With your literature prepared, you're ready to configure and execute the systematic review.

\begin{figure}[h]
    \centering
    \includegraphics[width=\textwidth]{figures/info_extract_tools.png}
    \caption{The prismAId Review tool processes text documents through configured LLM providers to extract structured information. The tool supports multiple LLM providers, ensemble reviews, and various output formats for comprehensive data extraction.}
    \label{fig:review_tools}
\end{figure}

\subsection{Finalizing Review Configuration}
\tip{When extracting numerical data (like percentages or measurements), use empty value arrays `[""]` to allow the model to extract the exact values rather than categorizing them.}
\begin{enumerate}
    \item Define your prompt structure to instruct the LLM.
    \item Specify the review items (information to extract) in your configuration.
\end{enumerate}

\begin{configbox}[Configuration: Prompt Structure]
\begin{lstlisting}[language=TOML]
[prompt]
persona = "You are an expert agricultural scientist conducting a systematic review on climate change impacts."
task = "Extract specific information about how climate change affects crop yields from the scientific paper text provided."
expected_result = "You should output a JSON object with key findings on crop types, climate factors, and measured impacts."
definitions = "'Crop yield' refers to the quantity of agricultural output harvested per unit of land area."
example = "For example, if the paper states 'wheat yields decreased by 5.2% per degree Celsius increase', report 'wheat' as crop_type, 'temperature increase' as climate_factor, and '-5.2% per °C' as yield_impact."
failsafe = "If information on a specific field is not provided in the document, respond with an empty string value."
\end{lstlisting}
\end{configbox}

\begin{configbox}[Configuration: Review Items Structure]
\begin{lstlisting}[language=TOML]
[review.1]
key = "crop_type"
values = ["wheat", "rice", "maize", "soybean", "barley", "other", "multiple", ""]

[review.2]
key = "climate_factor"
values = ["temperature increase", "precipitation change", "extreme weather", "CO2 levels", "multiple factors", "other", ""]

[review.3]
key = "yield_impact"
values = [""]

[review.4]
key = "adaptation_strategies_discussed"
values = ["yes", "no"]

[review.5]
key = "study_timeframe"
values = ["historical", "current", "future projection", "mixed", ""]
\end{lstlisting}
\end{configbox}

\subsection{Executing the Review}
\note{For large reviews, consider running a test on a small subset of papers first to validate your configuration before processing all documents.}
\begin{commandbox}[Command: Running the Systematic Review]
\begin{lstlisting}[language=Bash]
# Navigate to your project directory
cd my_systematic_review

# Run the review with your configuration file
./prismaid -project config/climate_yield_review.toml
\end{lstlisting}
\end{commandbox}

\subsection{Monitoring Progress}

\begin{enumerate}
    \item Monitor the console output for progress updates.\warning{Long-running reviews may encounter API rate limits. Use the `tpm\_limit` and `rpm\_limit` settings in your configuration to manage API usage and avoid interruptions.}
    \item Check log files if you've enabled detailed logging.
    \item Address any errors or warnings that appear during processing.
\end{enumerate}

\begin{commandbox}[Command: Running with Detailed Logging]
\begin{lstlisting}[language=Bash]
# First modify your config file to set log_level to "high"
# Then run the review
./prismaid -project config/climate_yield_review.toml
\end{lstlisting}
\end{commandbox}

\section{Interpreting Results}

Once your review is complete, you'll need to analyze and understand the extracted information.

\subsection{Understanding Output Files}

\begin{enumerate}
    \item Locate your results file (CSV or JSON format).\reminder{If you enabled `cot\_justification`, each paper will have a corresponding justification file that explains the reasoning behind the extracted information. These can be invaluable for validation and deeper understanding.}
    \item Open supplementary files like justifications or summaries if enabled.
    \item Understand the structure of the output data.
\end{enumerate}

\begin{infobox}[Example CSV Result Structure]
\begin{lstlisting}
filename,crop_type,climate_factor,yield_impact,adaptation_strategies_discussed,study_timeframe
paper1.txt,wheat,temperature increase,-6.4% per °C,yes,future projection
paper2.txt,rice,precipitation change,-3.2% per 10% rainfall decrease,yes,historical
paper3.txt,multiple,multiple factors,varied by region and crop,yes,mixed
\end{lstlisting}
\end{infobox}


\subsection{Validating Extraction Quality}

\begin{enumerate}
    \item Randomly select a subset of papers for manual validation.
    \item Compare the extracted information with the original papers.
    \tip{Aim to manually validate at least 10-20\% of your papers to ensure extraction quality. Focus particularly on papers with unusual or unexpected results.}

    \item Note any discrepancies or patterns in extraction errors.
\end{enumerate}

\subsection{Analyzing Patterns and Trends}

\begin{enumerate}
    \item Import your results into analysis software (R, Python, Excel, etc.).
    \item Perform descriptive statistics on your extracted data.
    \item Identify patterns, relationships, and outliers.
\end{enumerate}

\note{Extracting numerical data from free-text fields like `yield\_impact` may require additional processing before quantitative analysis.}
\begin{commandbox}[Example R Code for Basic Analysis]
\begin{lstlisting}[language=R]
# Load libraries
library(tidyverse)

# Read results
results <- read_csv("results/findings.csv")

# Basic summary
summary(results)

# Count papers by crop type
results %>%
  count(crop_type) %>%
  arrange(desc(n))

# Analyze yield impacts by climate factor
results %>%
  filter(climate_factor %in% c("temperature increase", "precipitation change")) %>%
  group_by(climate_factor, crop_type) %>%
  summarize(n_studies = n()) %>%
  arrange(desc(n_studies))
\end{lstlisting}
\end{commandbox}


\section{Exporting Findings}

The final step is to prepare your findings for presentation, publication, or further analysis.

\subsection{Generating Summary Tables}

\begin{enumerate}
    \item Create summary tables of key findings.
    \item Format the data appropriately for your target audience.\tip{Include both raw data and processed summary tables in your publications to enhance transparency and reproducibility.}
    \item Include relevant metadata about your review process.
\end{enumerate}

\begin{commandbox}[Example Python Code for Table Generation]
\begin{lstlisting}[language=Python]
import pandas as pd
import matplotlib.pyplot as plt

# Load results
results = pd.read_csv("results/findings.csv")

# Create pivot table
pivot = pd.pivot_table(
    results,
    index="crop_type",
    columns="climate_factor",
    values="filename",
    aggfunc="count",
    fill_value=0
)

# Save summary table
pivot.to_csv("results/summary_table.csv")
pivot.to_excel("results/summary_table.xlsx")

# Create visualization
pivot.plot(kind="bar", figsize=(12, 8))
plt.title("Number of Studies by Crop Type and Climate Factor")
plt.tight_layout()
plt.savefig("results/crop_climate_summary.png", dpi=300)
\end{lstlisting}
\end{commandbox}

\subsection{Documenting Your Methodology}

\reminder{For publication, include your full \texttt{prismAId} configuration file (with API keys removed) in supplementary materials to enhance reproducibility.}
\begin{enumerate}
    \item Document your complete review methodology.
    \item Include details about:
    \begin{itemize}
        \item Search strategy and sources
        \item Inclusion/exclusion criteria
        \item \texttt{prismAId} configuration and settings
        \item Validation procedures
        \item Analysis methods
    \end{itemize}
    \item Save your configuration files alongside your results.
\end{enumerate}

\subsection{Visualizing Results}

\warning{Be cautious about drawing causal conclusions from your systematic review findings, especially when AI-assisted tools are used for extraction. Clearly acknowledge limitations in your reporting.}
\begin{enumerate}
    \item Create appropriate visualizations based on your data type.
    \item Consider common visualization types:
    \begin{itemize}
        \item Bar charts for categorical comparisons
        \item Forest plots for effect sizes
        \item Heat maps for multidimensional relationships
        \item Network diagrams for concept relationships
    \end{itemize}
    \item Ensure visualizations accurately represent your findings.
\end{enumerate}

\subsection{Preparing for Publication}

\begin{enumerate}
    \item Format your findings according to journal or conference requirements.
    \item Follow PRISMA or other appropriate reporting guidelines.
    \item Create a PRISMA flow diagram documenting the review process.
    \item Properly cite \texttt{prismAId} in your methodology section.\note{Some journals may have specific requirements or guidelines for reporting AI-assisted analyses. Check with your target journal for any specific policies.}
\end{enumerate}

\begin{infobox}[Example Citation for prismAId]
\begin{lstlisting}
For the systematic information extraction, we used prismAId (Boero, 2024),
an open-source AI-assisted systematic review toolkit. The review was
conducted using [model name] with a temperature setting of [value] and
validation was performed on [percentage]% of the included studies.

Reference:
Boero, R. (2024). prismAId - Open Science AI Tools for Systematic,
Protocol-Based Literature Reviews. Zenodo.
https://doi.org/10.5281/zenodo.11210796
\end{lstlisting}
\end{infobox}

\section{Conclusion}

You have now completed a systematic review using \texttt{prismAId}. By following this step-by-step process, you've:

\begin{itemize}
    \item Set up a structured, reproducible review project
    \item Gathered and prepared literature systematically
    \item Configured and executed an AI-assisted information extraction
    \item Analyzed and validated the extracted information
    \item Prepared your findings for dissemination
\end{itemize}

\reminder{Systematic reviews are iterative processes. Based on your findings, you may decide to refine your research question, adjust your extraction parameters, or expand your literature search. \texttt{prismAId} makes these iterations more efficient than traditional methods.}

The combination of human expertise and AI assistance enables more comprehensive and efficient reviews while maintaining methodological rigor. The next chapter will explore advanced features of \texttt{prismAId} that can further enhance your systematic review capabilities.


% ==============================
% Part 4: Advanced Features
% ==============================
\part{Advanced Features}
\chapter[Ensemble Reviews]{Advanced Features: Ensemble Reviews} \label{chap:advanced_ensemble}

This chapter explores ensemble reviews, one of the most powerful advanced features of \texttt{prismAId}, enabling users to leverage multiple AI models simultaneously for enhanced reliability and insight.

\section{Understanding Ensemble Reviews}

Ensemble reviews represent a sophisticated approach to systematic literature analysis where multiple large language models (LLMs) are used to extract information from the same document set. This methodology parallels the traditional practice of having multiple human reviewers assess the same literature.

\subsection{Concept and Benefits}\tip{Ensemble reviews are particularly valuable when conducting high-stakes systematic reviews where accuracy and reliability are critical, such as in clinical guideline development or policy formation.}

\begin{itemize}
    \item \textbf{Definition}: An ensemble review utilizes two or more AI models, either from the same provider or across different providers, to independently extract information from each document.
    \item \textbf{Theoretical Foundation}: The approach is grounded in ensemble learning theory, where multiple algorithms collectively produce more accurate and robust results than individual algorithms.
\end{itemize}

\begin{configbox}[Configuration: Simple Ensemble with Two Models]
\begin{lstlisting}[language=TOML]
[project.llm.1]
provider = "OpenAI"
api_key = ""
model = "gpt-4o-mini"
temperature = 0.01
tpm_limit = 0
rpm_limit = 0

[project.llm.2]
provider = "OpenAI"
api_key = ""
model = "gpt-4o"
temperature = 0.01
tpm_limit = 0
rpm_limit = 0
\end{lstlisting}
\end{configbox}

\subsection{Key Advantages}

Ensemble reviews offer several significant benefits:

\begin{itemize}
    \item \textbf{Uncertainty Quantification}: Variations in model outputs highlight where information may be ambiguous or open to interpretation.
    \item \textbf{Increased Confidence}: Strong agreement across models suggests higher reliability of the extracted information.
    \item \textbf{Bias Reduction}: Different models may have different biases; using multiple models helps identify and mitigate these biases.
    \item \textbf{Robustness to Model Limitations}: Different models excel at different tasks; using multiple models compensates for individual weaknesses.
\end{itemize}

\section{Configuring Multi-Model Ensembles}

\texttt{prismAId} offers exceptional flexibility in configuring ensemble reviews, allowing you to combine models from the same provider, different providers, or a mix of both.

\subsection{Using Multiple Models from a Single Provider}

This approach allows you to compare different models with varying capabilities from the same provider:\tip{When using multiple models from the same provider, consider including a range of model sizes to balance cost, speed, and accuracy. For example, include both gpt-3.5-turbo and gpt-4o to compare a faster, more economical model with a more sophisticated one.}

\begin{configbox}[Configuration: Ensemble with Multiple OpenAI Models]
\begin{lstlisting}[language=TOML]
[project.llm.1]
provider = "OpenAI"
api_key = ""
model = "gpt-3.5-turbo"
temperature = 0.01
tpm_limit = 0
rpm_limit = 0

[project.llm.2]
provider = "OpenAI"
api_key = ""
model = "gpt-4o-mini"
temperature = 0.01
tpm_limit = 0
rpm_limit = 0

[project.llm.3]
provider = "OpenAI"
api_key = ""
model = "gpt-4o"
temperature = 0.01
tpm_limit = 0
rpm_limit = 0
\end{lstlisting}
\end{configbox}

\subsection{Cross-Provider Ensemble Configuration}

Using models from different providers can be particularly valuable, as these models may have been trained on different datasets and using different architectures:\warning{When using models from multiple providers, ensure you have valid API keys for each provider configured either in your TOML file or as environment variables. Missing or invalid keys will cause the review to fail for that provider.}

\begin{configbox}[Configuration: Cross-Provider Ensemble Example]
\begin{lstlisting}[language=TOML]
[project.llm.1]
provider = "OpenAI"
api_key = ""
model = "gpt-4o-mini"
temperature = 0.01
tpm_limit = 0
rpm_limit = 0

[project.llm.2]
provider = "GoogleAI"
api_key = ""
model = "gemini-1.5-flash"
temperature = 0.01
tpm_limit = 0
rpm_limit = 0

[project.llm.3]
provider = "Anthropic"
api_key = ""
model = "claude-3-haiku"
temperature = 0.01
tpm_limit = 0
rpm_limit = 0
\end{lstlisting}
\end{configbox}

\subsection{Comprehensive Five-Provider Ensemble}

For maximum diversity and robustness, you can configure an ensemble using models from all supported providers:\reminder{Pay close attention to the temperature settings in ensemble reviews. Lower temperatures (0.01-0.1) produce more deterministic outputs, making it easier to compare responses across models.}


\begin{configbox}[Configuration: Full Five-Provider Ensemble]
\begin{lstlisting}[language=TOML]
[project.llm.1]
provider = "OpenAI"
api_key = ""
model = "gpt-4o-mini"
temperature = 0.01
tpm_limit = 0
rpm_limit = 0

[project.llm.2]
provider = "GoogleAI"
api_key = ""
model = "gemini-1.5-flash"
temperature = 0.01
tpm_limit = 0
rpm_limit = 0

[project.llm.3]
provider = "Cohere"
api_key = ""
model = "command-r"
temperature = 0.01
tpm_limit = 0
rpm_limit = 0

[project.llm.4]
provider = "Anthropic"
api_key = ""
model = "claude-3-haiku"
temperature = 0.01
tpm_limit = 0
rpm_limit = 0

[project.llm.5]
provider = "DeepSeek"
api_key = ""
model = "deepseek-chat"
temperature = 0.01
tpm_limit = 0
rpm_limit = 0
\end{lstlisting}
\end{configbox}

\section{Analyzing Ensemble Results}

After running an ensemble review, you'll need specialized approaches to analyze and interpret the results from multiple models.

\subsection{Output Format and Structure}

When running an ensemble review, \texttt{prismAId} creates separate result files for each model used:

\begin{infobox}[Example Output Files from Ensemble Review]
\begin{lstlisting}
results_OpenAI_gpt-4o-mini.csv
results_GoogleAI_gemini-1.5-flash.csv
results_Anthropic_claude-3-haiku.csv
\end{lstlisting}
\end{infobox}

The individual model files contain the standard extraction results. Analyzing them, researchers whould prepare ensemble summary file containing consensus metrics and comparison data.

\subsection{Quantifying Model Agreement}

To analyze the level of agreement between models:\tip{For categorical variables, Cohen's kappa or Fleiss' kappa (for more than two models) can provide a more sophisticated measure of inter-model agreement than simple percentage agreement.}

\begin{commandbox}[R Code for Analyzing Model Agreement]
\begin{lstlisting}[language=R]
library(tidyverse)

# Load results from different models
model1 <- read_csv("results_OpenAI_gpt-4o-mini.csv")
model2 <- read_csv("results_GoogleAI_gemini-1.5-flash.csv")
model3 <- read_csv("results_Anthropic_claude-3-haiku.csv")

# Join datasets
comparison <- model1 %>%
  select(filename, crop_type, climate_factor) %>%
  rename(crop_type_model1 = crop_type,
         climate_factor_model1 = climate_factor) %>%
  left_join(
    model2 %>%
      select(filename, crop_type, climate_factor) %>%
      rename(crop_type_model2 = crop_type,
             climate_factor_model2 = climate_factor),
    by = "filename"
  ) %>%
  left_join(
    model3 %>%
      select(filename, crop_type, climate_factor) %>%
      rename(crop_type_model3 = crop_type,
             climate_factor_model3 = climate_factor),
    by = "filename"
  )

# Calculate agreement for categorical variables
comparison <- comparison %>%
  mutate(
    crop_type_agreement = case_when(
      crop_type_model1 == crop_type_model2 & crop_type_model2 == crop_type_model3 ~ "full_agreement",
      crop_type_model1 == crop_type_model2 | crop_type_model1 == crop_type_model3 | crop_type_model2 == crop_type_model3 ~ "partial_agreement",
      TRUE ~ "no_agreement"
    ),
    climate_factor_agreement = case_when(
      climate_factor_model1 == climate_factor_model2 & climate_factor_model2 == climate_factor_model3 ~ "full_agreement",
      climate_factor_model1 == climate_factor_model2 | climate_factor_model1 == climate_factor_model3 | climate_factor_model2 == climate_factor_model3 ~ "partial_agreement",
      TRUE ~ "no_agreement"
    )
  )

# Summarize agreement levels
agreement_summary <- comparison %>%
  summarize(
    crop_type_full_agreement = mean(crop_type_agreement == "full_agreement"),
    crop_type_partial_agreement = mean(crop_type_agreement == "partial_agreement"),
    crop_type_no_agreement = mean(crop_type_agreement == "no_agreement"),
    climate_factor_full_agreement = mean(climate_factor_agreement == "full_agreement"),
    climate_factor_partial_agreement = mean(climate_factor_agreement == "partial_agreement"),
    climate_factor_no_agreement = mean(climate_factor_agreement == "no_agreement")
  )

print(agreement_summary)
\end{lstlisting}
\end{commandbox}


\subsection{Visualizing Ensemble Results}

Visualizations can help identify patterns of agreement and disagreement:\note{For numerical data extraction (like percentages or measurements), consider using scatterplots to visualize the correlation between values extracted by different models, or boxplots to show the distribution of values across models.}

\begin{commandbox}[Python Code for Visualizing Ensemble Agreement]
\begin{lstlisting}[language=Python]
import pandas as pd
import matplotlib.pyplot as plt
import seaborn as sns

# Load results from different models
model1 = pd.read_csv("results_OpenAI_gpt-4o-mini.csv")
model2 = pd.read_csv("results_GoogleAI_gemini-1.5-flash.csv")
model3 = pd.read_csv("results_Anthropic_claude-3-haiku.csv")

# Create comparison dataframe for a specific variable
comparison = pd.DataFrame({
    'filename': model1['filename'],
    'OpenAI': model1['crop_type'],
    'GoogleAI': model2['crop_type'],
    'Anthropic': model3['crop_type']
})

# Count agreements for each paper
comparison['agreement_count'] = comparison.apply(
    lambda row: len(set([row['OpenAI'], row['GoogleAI'], row['Anthropic']])),
    axis=1
)

# Create a heatmap of disagreements
# Convert to wide format with models as columns and papers as rows
heatmap_data = comparison.set_index('filename')[['OpenAI', 'GoogleAI', 'Anthropic']]

# Create categorical mapping for visualization
unique_values = sorted(list(set(
    heatmap_data['OpenAI'].tolist() +
    heatmap_data['GoogleAI'].tolist() +
    heatmap_data['Anthropic'].tolist()
)))
value_map = {value: i for i, value in enumerate(unique_values)}

# Apply mapping
for col in heatmap_data.columns:
    heatmap_data[col] = heatmap_data[col].map(value_map)

# Create heatmap
plt.figure(figsize=(12, len(heatmap_data) * 0.3))
sns.heatmap(heatmap_data, cmap='viridis', yticklabels=True)
plt.title('Model Agreement Visualization (Crop Type)')
plt.tight_layout()
plt.savefig('ensemble_agreement_heatmap.png', dpi=300)

# Create agreement distribution chart
agreement_counts = comparison['agreement_count'].value_counts().sort_index()
plt.figure(figsize=(10, 6))
agreement_counts.plot(kind='bar')
plt.title('Distribution of Model Agreement')
plt.xlabel('Number of Unique Answers (1 = Full Agreement)')
plt.ylabel('Number of Papers')
plt.tight_layout()
plt.savefig('agreement_distribution.png', dpi=300)
\end{lstlisting}
\end{commandbox}


\subsection{Decision Strategies for Conflicting Results}

When models disagree on extracted information, several strategies can be employed:\warning{Be cautious when using majority voting for numerical values, as this may lead to misleading results. For numerical data, consider using the median or mean, and examine the standard deviation to identify high-variance cases.}

\begin{itemize}
    \item \textbf{Majority Voting}: Use the value that the majority of models agree on.
    \item \textbf{Weighted Voting}: Assign higher weight to more capable models (e.g., GPT-4 over GPT-3.5).
    \item \textbf{Conservative Approach}: For papers with significant disagreement, flag for manual review.
    \item \textbf{Model-Specific Trust}: For certain types of information, one model may consistently outperform others.
\end{itemize}

\begin{commandbox}[Python Code for Majority Voting]
\begin{lstlisting}[language=Python]
import pandas as pd
from collections import Counter

# Load results from different models
model1 = pd.read_csv("results_OpenAI_gpt-4o-mini.csv")
model2 = pd.read_csv("results_GoogleAI_gemini-1.5-flash.csv")
model3 = pd.read_csv("results_Anthropic_claude-3-haiku.csv")

# Prepare a consolidated results dataframe
consolidated = pd.DataFrame({'filename': model1['filename']})

# Apply majority voting for each extracted field
for field in ['crop_type', 'climate_factor', 'adaptation_strategies_discussed']:
    consolidated[field] = [
        Counter([m1, m2, m3]).most_common(1)[0][0]
        for m1, m2, m3 in zip(
            model1[field].fillna(''),
            model2[field].fillna(''),
            model3[field].fillna('')
        )
    ]

    # For fields where all models disagree, mark for review
    consolidated[f'{field}_review_needed'] = [
        len(set([m1, m2, m3])) == 3
        for m1, m2, m3 in zip(
            model1[field].fillna(''),
            model2[field].fillna(''),
            model3[field].fillna('')
        )
    ]

# For numerical fields, take the median
for field in ['yield_impact_percentage']:
    try:
        # Convert to numeric, errors='coerce' will convert non-numeric to NaN
        vals1 = pd.to_numeric(model1[field], errors='coerce')
        vals2 = pd.to_numeric(model2[field], errors='coerce')
        vals3 = pd.to_numeric(model3[field], errors='coerce')

        # Calculate median
        consolidated[field] = [
            pd.Series([v1, v2, v3]).median()
            for v1, v2, v3 in zip(vals1, vals2, vals3)
        ]

        # Calculate standard deviation to identify high variance
        consolidated[f'{field}_std'] = [
            pd.Series([v1, v2, v3]).std()
            for v1, v2, v3 in zip(vals1, vals2, vals3)
        ]

        # Flag for review if standard deviation is high
        consolidated[f'{field}_review_needed'] = consolidated[f'{field}_std'] > 2.0
    except:
        # Handle cases where the field might not exist or can't be processed
        pass

# Save consolidated results
consolidated.to_csv("results_consolidated.csv", index=False)
\end{lstlisting}
\end{commandbox}

\section{Case Studies in Ensemble Reviews}

To illustrate the practical application of ensemble reviews, let's examine some case scenarios and best practices.

\subsection{Case 1: Basic Verification Ensemble}

\begin{configbox}[Configuration: Verification Ensemble]
\begin{lstlisting}[language=TOML]
# A simple two-model ensemble using different providers
# for basic verification of extraction results

[project.llm.1]
provider = "OpenAI"
api_key = ""
model = "gpt-3.5-turbo"  # More economical option
temperature = 0.01
tpm_limit = 0
rpm_limit = 0

[project.llm.2]
provider = "Anthropic"
api_key = ""
model = "claude-3-haiku"  # Similar capability level
temperature = 0.01
tpm_limit = 0
rpm_limit = 0
\end{lstlisting}
\end{configbox}

\textbf{Use Case}: This configuration is suitable for routine systematic reviews where you want a basic cross-check between providers without significantly increasing costs. It uses comparably efficient models from different providers.

\textbf{Expected Outcome}: When these models agree, you can have higher confidence in the extraction. When they disagree, you might flag those specific data points for manual review.

\subsection{Case 2: Tiered Capability Ensemble}

\begin{configbox}[Configuration: Tiered Capability Ensemble]
\begin{lstlisting}[language=TOML]
# A three-model ensemble using progressively more capable models
# from the same provider for comparative analysis

[project.llm.1]
provider = "OpenAI"
api_key = ""
model = "gpt-3.5-turbo"  # Base level
temperature = 0.01
tpm_limit = 0
rpm_limit = 0

[project.llm.2]
provider = "OpenAI"
api_key = ""
model = "gpt-4o-mini"  # Mid level
temperature = 0.01
tpm_limit = 0
rpm_limit = 0

[project.llm.3]
provider = "OpenAI"
api_key = ""
model = "gpt-4o"  # Top level
temperature = 0.01
tpm_limit = 0
rpm_limit = 0
\end{lstlisting}
\end{configbox}

\textbf{Use Case}: This approach is useful for understanding how model capability affects extraction quality. It can help determine if investing in more advanced models provides meaningful improvements for your specific extraction tasks.

\textbf{Expected Outcome}: By comparing results across models of increasing capability, you can determine the minimum model level needed for reliable extraction, potentially optimizing future costs.

\subsection{Case 3: Comprehensive Cross-Provider Ensemble}

\begin{configbox}[Configuration: Comprehensive Cross-Provider Ensemble]
\begin{lstlisting}[language=TOML]
# A five-provider ensemble using top-tier models from each provider
# for maximum reliability in high-stakes reviews

[project.llm.1]
provider = "OpenAI"
api_key = ""
model = "gpt-4o"
temperature = 0.01
tpm_limit = 0
rpm_limit = 0

[project.llm.2]
provider = "GoogleAI"
api_key = ""
model = "gemini-1.5-pro"
temperature = 0.01
tpm_limit = 0
rpm_limit = 0

[project.llm.3]
provider = "Cohere"
api_key = ""
model = "command-r-plus"
temperature = 0.01
tpm_limit = 0
rpm_limit = 0

[project.llm.4]
provider = "Anthropic"
api_key = ""
model = "claude-3-opus"
temperature = 0.01
tpm_limit = 0
rpm_limit = 0

[project.llm.5]
provider = "DeepSeek"
api_key = ""
model = "deepseek-chat"
temperature = 0.01
tpm_limit = 0
rpm_limit = 0
\end{lstlisting}
\end{configbox}

\textbf{Use Case}: This configuration is ideal for high-stakes systematic reviews such as those informing clinical guidelines, policy decisions, or major meta-analyses where maximum reliability is essential.\tip{Start with a small test dataset when using comprehensive ensembles like this to estimate costs and review time before committing to your full dataset.}

\textbf{Expected Outcome}: This approach provides the highest confidence in results through diverse model architectures and training data. Points of unanimous agreement across all five providers suggest very high reliability.


\subsection{Best Practices for Ensemble Reviews}

Based on case studies and practical experience, here are recommended best practices:\reminder{When reporting results from ensemble reviews in publications, clearly state which models were used, how disagreements were resolved, and what level of agreement was observed for key findings.}

\begin{itemize}
    \item \textbf{Start Small}: Test your ensemble configuration on a small subset (5-10 papers) to identify any issues before running the full review.
    \item \textbf{Match Complexity to Need}: Use simpler ensembles for routine reviews and more comprehensive ensembles for high-stakes reviews.
    \item \textbf{Control Variables}: Keep temperature settings consistent across models to ensure fair comparisons.
    \item \textbf{Budget Appropriately}: Ensemble reviews will multiply your API costs by the number of models used; plan accordingly.
    \item \textbf{Document Model Versions}: Note the specific model versions used, as providers regularly update their models.
    \item \textbf{Analyze Disagreements}: Patterns in model disagreements often reveal ambiguities in your prompt or in the literature itself.
\end{itemize}

\chapter[Debugging, Costs \& Integration]{Advanced Features: Debugging, Cost Management \& Integration} \label{chap:advanced_features}

Building on our exploration of ensemble reviews, this chapter covers additional advanced features of \texttt{prismAId} that can enhance your systematic review process: debugging techniques, cost management strategies, advanced prompt engineering, and workflow integration.

\section{Debugging and Validation Techniques}

\texttt{prismAId} offers several advanced debugging features that can help identify and resolve issues in your systematic review process.

\subsection{Log Levels and Debugging Output}

The \texttt{log\_level} parameter in your configuration file controls the verbosity of debugging information:

\begin{configbox}[Configuration: Log Level Settings]
\begin{lstlisting}[language=TOML]
[project.configuration]
log_level = "high"  # Options: "low", "medium", "high"
\end{lstlisting}
\end{configbox}

Each level provides different information:\tip{When troubleshooting issues, start by setting the log level to "medium" to get more detailed console output. If the issue persists, switch to "high" for complete logging to file.}
\begin{itemize}
    \item \textbf{Low}: Minimal information, only essential status updates
    \item \textbf{Medium}: Detailed process information printed to the console
    \item \textbf{High}: Comprehensive logs saved to a file, including all API interactions
\end{itemize}

\begin{commandbox}[Command: Running with High Log Level]
\begin{lstlisting}[language=Bash]
# First set log_level to "high" in your config file
./prismaid -project your_project.toml

# Check the log file created in your project directory
cat your_project.log
\end{lstlisting}
\end{commandbox}

\subsection{Duplication for Consistency Testing}

The duplication feature runs the same review twice on identical inputs, allowing you to assess the consistency of model responses:\warning{Enabling duplication will double your API usage and costs. Use this feature selectively for testing or when consistency validation is critical.}

\begin{configbox}[Configuration: Enabling Duplication]
\begin{lstlisting}[language=TOML]
[project.configuration]
duplication = "yes"  # Options: "yes", "no"
\end{lstlisting}
\end{configbox}

The duplication process:
\begin{enumerate}
    \item Creates temporary copies of your input files
    \item Processes them as a separate batch
    \item Compares results for consistency
    \item Cleans up temporary files
\end{enumerate}

\subsection{Chain-of-Thought Justification}

The Chain-of-Thought (CoT) justification feature provides visibility into the model's reasoning process:

\begin{configbox}[Configuration: Enabling CoT Justification]
\begin{lstlisting}[language=TOML]
[project.configuration]
cot_justification = "yes"  # Options: "yes", "no"
\end{lstlisting}
\end{configbox}

When enabled, this creates a separate .txt file for each processed document containing:\tip{CoT justifications are invaluable for validating extraction quality, identifying potential misinterpretations, and understanding why models might be struggling with particular papers or information types.}
\begin{itemize}
    \item The model's reasoning for each extracted data point
    \item Key passages from the document that influenced the extraction
    \item Any uncertainty or alternative interpretations considered
\end{itemize}


\section{Cost and Rate Management}

\texttt{prismAId} provides several features to help manage costs and API rate limits.

\subsection{Token and Request Rate Limiting}
\note{Each provider has different rate limit structures. OpenAI limits by TPM (tokens per minute), Anthropic by RPM (requests per minute), and others may have tiered systems. Check the provider's documentation for current limits.}

\begin{configbox}[Configuration: Rate Limiting Settings]
\begin{lstlisting}[language=TOML]
[project.llm.1]
provider = "OpenAI"
api_key = ""
model = "gpt-4o-mini"
temperature = 0.01
tpm_limit = 100000  # Tokens per minute limit
rpm_limit = 60      # Requests per minute limit
\end{lstlisting}
\end{configbox}

These settings help you:
\begin{itemize}
    \item \textbf{Avoid Provider Rate Limiting}: Stay below provider-enforced limits\warning{These limits can change as providers update their services. Always check the most recent documentation from each provider for current rate limits.}
    \item \textbf{Balance Resource Usage}: Prevent spikes in API consumption
\end{itemize}

\begin{infobox}[Example Provider Rate Limits]
\begin{lstlisting}
Provider  | Model            | Default TPM  | Default RPM
----------|------------------|--------------|------------
OpenAI    | gpt-3.5-turbo    | 60,000       | 3,500
OpenAI    | gpt-4o           | 10,000       | 500
Anthropic | claude-3-haiku   | N/A          | 5,000
GoogleAI  | gemini-1.5-flash | 4,000,000    | N/A
Cohere    | command-r        | 100,000      | 1,000
\end{lstlisting}
\end{infobox}

Please note that not respecting rate limits may result in API errors and in not obtaining results for some manuscripts.

\subsection{Automatic Cost Minimization}

\texttt{prismAId} can automatically select the most cost-effective model by leaving the model field empty:\tip{Automatic model selection is particularly useful for heterogeneous document collections where some papers may be too large for smaller models.}

\begin{configbox}[Configuration: Cost Minimization]
\begin{lstlisting}[language=TOML]
[project.llm.1]
provider = "OpenAI"
api_key = ""
model = ""  # Empty value enables automatic model selection
temperature = 0.01
tpm_limit = 0
rpm_limit = 0
\end{lstlisting}
\end{configbox}

How automatic model selection works:
\begin{enumerate}
    \item \texttt{prismAId} calculates the token count for each document
    \item It determines which provider models can handle the document within token limits
    \item Among eligible models, it selects the most cost-efficient option
    \item Different documents may be processed by different models based on size
\end{enumerate}

\section{Advanced Prompt Engineering}

The quality of your systematic review results largely depends on how effectively you structure your prompts. Here are advanced strategies for optimizing extraction through prompt engineering.

\subsection{Optimizing Prompt Components}

Each component of the prompt structure serves a specific purpose and can be optimized:

\begin{configbox}[Configuration: Advanced Prompt Components]
\begin{lstlisting}[language=TOML]
[prompt]
persona = "You are an expert environmental scientist specializing in climate change impacts on agriculture, with experience conducting systematic reviews according to PRISMA guidelines."
task = "Analyze the scientific paper provided and extract specific information about climate change impacts on crop yields, methodology used, and adaptation strategies discussed."
expected_result = "You should output a JSON object containing values for each key specified in the review structure. Values must adhere exactly to the prescribed formats and vocabulary."
definitions = "'Crop yield' refers to harvestable production per unit of land area. 'Statistical significance' means p < 0.05 unless otherwise specified in the paper. 'Adaptation strategies' include any interventions meant to reduce negative climate impacts on agriculture."
example = "For instance, if the paper states 'wheat yields decreased by 5.2% (p < 0.01) per degree Celsius warming, with irrigation mitigating 40% of losses', you would extract: {\"crop_type\": \"wheat\", \"climate_factor\": \"temperature increase\", \"yield_impact\": \"-5.2% per °C\", \"statistical_significance\": \"yes\", \"adaptation_strategies_discussed\": \"yes\", \"adaptation_effectiveness\": \"40% loss reduction\"}"
failsafe = "If specific information is not clearly stated in the document, do not speculate or infer beyond reasonable scientific interpretation. Respond with an empty string for missing data. If the paper doesn't address the topic at all, use 'not addressed' for categorical fields."
\end{lstlisting}
\end{configbox}

Key strategies for each component:\note{Research suggests that more detailed and domain-specific prompts generally produce more accurate extractions, particularly for specialized scientific concepts.}

\begin{itemize}
    \item \textbf{Persona}: Include relevant expertise and methodological background to establish appropriate context.
    \item \textbf{Task}: Be specific about analytical depth and the nature of extraction required.
    \item \textbf{Expected Result}: Define the exact output format and emphasize adherence to specified value constraints.
    \item \textbf{Definitions}: Provide domain-specific terminology explanations, especially for potentially ambiguous concepts.
    \item \textbf{Example}: Show realistic examples that cover different data patterns and edge cases.
    \item \textbf{Failsafe}: Include clear guidelines for handling uncertainty, ambiguity, and missing information.
\end{itemize}


\subsection{Advanced Review Structure Patterns}

Beyond context setting above, you may implement more sophisticated patterns in your review structure:

\begin{configbox}[Configuration: Advanced Review Structure Patterns]
\begin{lstlisting}[language=TOML]
# Hierarchical extraction
[review.1]
key = "methodology_type"
values = ["experimental", "observational", "modeling", "review", "mixed", "other", ""]

[review.2]
key = "experimental_design"
values = ["randomized controlled trial", "non-randomized controlled trial", "before-after", "other", "not applicable", ""]

# Conditional extraction
[review.3]
key = "sample_size_reported"
values = ["yes", "no"]

[review.4]
key = "sample_size_value"
values = [""]

# Relational extraction
[review.5]
key = "crop_types_studied"
values = [""]

[review.6]
key = "primary_crop"
values = [""]

# Confidence assessment
[review.7]
key = "statistical_methods_quality"
values = ["high", "medium", "low", "not applicable", "cannot determine", ""]

\end{lstlisting}
\end{configbox}

Advanced patterns include:\tip{When using advanced patterns, ensure your prompt includes clear instructions on how these patterns relate to each other, particularly for conditional extractions.}

\begin{itemize}
    \item \textbf{Hierarchical Extraction}: Extract general categories first, then specific details based on those categories.
    \item \textbf{Conditional Extraction}: Use yes/no fields to establish presence, then extract specific values only if present.
    \item \textbf{Relational Extraction}: Capture relationships between different entities (e.g., primary crop among multiple crops studied).
    \item \textbf{Confidence Assessment}: Include quality assessments of methodological elements.
\end{itemize}


\subsection{Iterative Prompt Refinement}

The most effective prompts are developed through an iterative process:\warning{Changing prompts mid-review can introduce inconsistency. Once you begin your full review, avoid modifying the prompt unless absolutely necessary. If changes are required, consider re-processing previously analyzed papers.}

\begin{infobox}[Iterative Prompt Development Process]
\begin{lstlisting}
1. Start with a basic prompt and review structure
2. Test on 3-5 representative papers
3. Analyze results for:
   - Incorrect extractions
   - Missing information
   - Ambiguous responses
   - Inconsistent formatting
4. Identify patterns in errors or weaknesses
5. Refine prompt components and review structure
6. Test again on the same papers plus 2-3 new ones
7. Repeat until results match expectations
8. Document all prompt versions and their performance
\end{lstlisting}
\end{infobox}

For more complex extractions, consider dividing review items across multiple project configurations, allowing for more tailored context prompts with specific examples. While this approach can improve extraction accuracy, be aware that it increases costs as manuscripts must be processed multiple times.

\section{Workflow Integration and Automation}

\texttt{prismAId} can be integrated into broader research workflows and automated pipelines.\reminder{For very large reviews (hundreds or thousands of papers), consider implementing a database backend to store intermediate results and enable incremental processing.}

\subsection{Integration with Research Workflows}

\texttt{prismAId} can be integrated with broader research tools and workflows:

\begin{itemize}
    \item \textbf{Reference Management}: Integrate with Zotero, Mendeley, or EndNote for literature organization
    \item \textbf{Statistical Analysis}: Export results to R, SPSS, or specialized meta-analysis software
    \item \textbf{Collaborative Research}: Share configurations and results through version control systems
    \item \textbf{Publication Workflows}: Generate formatted tables and figures for manuscript inclusion
\end{itemize}

\section{Conclusion}

Advanced features of \texttt{prismAId} significantly enhance its capabilities beyond basic systematic reviews.\tip{Start with simpler configurations and gradually incorporate advanced features as you become more familiar with \texttt{prismAId}. Each feature adds power but also complexity, so build your expertise incrementally.} Ensemble reviews provide increased confidence and uncertainty quantification. Debugging features help identify and resolve issues in the extraction process. Cost management strategies optimize resource utilization, while advanced prompt engineering improves the quality and reliability of extractions. Finally, programmatic integration and automation enable seamless incorporation of \texttt{prismAId} into comprehensive research workflows.

By mastering these advanced features, researchers can conduct more sophisticated, reliable, and efficient systematic reviews that maintain the highest standards of methodological rigor while leveraging the power of artificial intelligence.


% ==============================
% Part 5: Troubleshooting \& FAQs
% ==============================
\part{Troubleshooting \& FAQs}
\chapter{Troubleshooting Common Issues} \label{chap:troubleshooting}
- Installation problems  
- Configuration errors  
- Common fixes  

\chapter{Frequently Asked Questions} \label{chap:faq}
- Can I use prismAId with [X] software?  
- How to optimize performance?  

\end{document}
