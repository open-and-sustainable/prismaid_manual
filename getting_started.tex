\chapter{Installation \& Setup}
\label{chap:install}

This chapter explains how to install \texttt{prismAId}, covering system requirements, installation methods, and first-time setup for each of the toolkit's components.

\section{System Requirements}

Before installing \texttt{prismAId}, ensure your system meets the following requirements:\warning{Ensure that you have administrative privileges on your system before installing \texttt{prismAId}, especially on Windows and macOS.}

\begin{itemize}
    \item \textbf{Operating System}: Windows 10 or later, macOS 11 or later, or a Linux distribution (Arch, Debian, Fedora, Ubuntu, etc.).
    \item \textbf{Processor}: 64-bit CPU (Intel, AMD, or ARM64).
    \item \textbf{Memory}: At least 4GB RAM (8GB recommended).
    \item \textbf{Storage}: Minimum 500MB of free disk space.
    \item \textbf{Internet Connection}: Required for downloading packages and accessing LLM APIs.
    \item \textbf{API Keys}: To use the Review tool, you'll need an API key from at least one of the supported LLM providers (OpenAI, GoogleAI, Cohere, Anthropic, or DeepSeek).
\end{itemize}

\section{Installation Methods}

\texttt{prismAId} can be installed in multiple ways, depending on your preferences and workflow requirements.\warning{The capitalization of the letters in the prismAId tool follows the conventions of each programming language, ensuring consistency within each context but resulting in variations across packages and languages.}
 Choose the method that best suits your needs.

\subsection{Method 1: Standalone Binaries}

The simplest installation method is to download the pre-compiled binaries, which require no additional dependencies.\note{Binaries are standalone executables that work on Linux, macOS, and Windows, supporting both AMD64 and Arm64 architectures.}

\begin{enumerate}
    \item Navigate to the \href{https://github.com/open-and-sustainable/prismaid/releases}{GitHub Releases page}.
    \item Download the appropriate binary for your operating system.
    \item Place the file into a directory of your choice.
    \item On Linux and macOS, set the binaries as executable files.
\end{enumerate}

\subsubsection{Windows}
\reminder{On Windows, you may need to allow execution if prompted by security settings.}
\begin{commandbox}[Command: Running \texttt{prismAId} on Windows]
After downloading, navigate to the folder and run:
\begin{lstlisting}[language=Bash]
# Download papers from Zotero
./prismaid.exe -download-zotero zotero_config.toml

# Download papers from URL list
./prismaid.exe -download-URL paper_urls.txt

# Convert PDF files to text
./prismaid.exe -convert-pdf ./papers

# Run the Review tool with a project configuration
./prismaid.exe -project your_project.toml
\end{lstlisting}
\end{commandbox}


\subsubsection{macOS}
\warning{You may need to approve the application in \textbf{System Preferences} under \textbf{Security \& Privacy} before running it.}
\begin{commandbox}[Command: Running \texttt{prismAId} on macOS]
After downloading, give execution permissions and run:
\begin{lstlisting}[language=Bash]
chmod +x prismaid

# Download papers from Zotero
./prismaid -download-zotero zotero_config.toml

# Convert various file formats to text
./prismaid -convert-pdf ./papers
./prismaid -convert-docx ./papers
./prismaid -convert-html ./papers

# Run the Review tool with a project configuration
./prismaid -project your_project.toml
\end{lstlisting}
\end{commandbox}

\subsubsection{Linux}
\tip{If you receive a "Permission denied" error, try running \texttt{chmod +x} again or executing with \texttt{sudo}.}
\begin{commandbox}[Command: Running \texttt{prismAId} on Linux]
\begin{lstlisting}[language=Bash]
chmod +x prismaid

# Initialize a new project configuration interactively
./prismaid -init

# Download papers from a URL list
./prismaid -download-URL paper_urls.txt

# Run the Review tool with a project configuration
./prismaid -project your_project.toml
\end{lstlisting}
\end{commandbox}


\subsection{Method 2: Go Package}

If you are a Go developer, you can use \texttt{prismAId} as a Go package.\tip{The Go package offers the most comprehensive functionality, as it is the native implementation.}

\begin{commandbox}[Command: Installing the Go Package]
\begin{lstlisting}[language=Bash]
go get "github.com/open-and-sustainable/prismaid"
\end{lstlisting}
\end{commandbox}

\begin{commandbox}[Code: Using prismAId in Go]
\begin{lstlisting}[language=Go]
import "github.com/open-and-sustainable/prismaid"

// Download papers from Zotero
err := prismaid.DownloadZoteroPDFs(username, apiKey, collectionName, parentDir)

// Download from URL list
err := prismaid.DownloadURLList("path/to/urls.txt")

// Convert files to text
err := prismaid.Convert(inputDir, "pdf,docx,html")

// Run a systematic review
err := prismaid.Review(tomlConfigString)
\end{lstlisting}
\end{commandbox}

\subsection{Method 3: Python Package}

For Python users, \texttt{prismAId} is available as a package on PyPI.\note{The Python package works on Linux and Windows AMD64, and macOS Arm64.}

\begin{commandbox}[Command: Installing the Python Package]
\begin{lstlisting}[language=Bash]
pip install prismaid
\end{lstlisting}
\end{commandbox}

\begin{commandbox}[Code: Using prismAId in Python]
\begin{lstlisting}[language=Python]
import prismaid

# Download papers from Zotero
prismaid.download_zotero_pdfs("username", "api_key", "collection_name", "./papers")

# Download from URL list
prismaid.download_url_list("urls.txt")

# Convert files to text
prismaid.convert("./papers", "pdf,docx,html")

# Run a systematic review
with open("project.toml", "r") as file:
    toml_config = file.read()
prismaid.review(toml_config)
\end{lstlisting}
\end{commandbox}

\subsection{Method 4: R Package}

For R users, \texttt{prismAId} is available as a package on R-universe.\note{The R package works on Linux AMD64 and macOS Arm64.}

\begin{commandbox}[Command: Installing the R Package]
\begin{lstlisting}[language=R]
install.packages("prismaid", repos = c("https://open-and-sustainable.r-universe.dev", "https://cloud.r-project.org"))
\end{lstlisting}
\end{commandbox}

\begin{commandbox}[Code: Using prismAId in R]
\begin{lstlisting}[language=R]
library(prismaid)

# Download papers from Zotero
DownloadZoteroPDFs("username", "api_key", "collection_name", "./papers")

# Download from URL list
DownloadURLList("urls.txt")

# Convert files to text
Convert("./papers", "pdf,docx,html")

# Run a systematic review
toml_content <- paste(readLines("project.toml"), collapse = "\n")
RunReview(toml_content)  # Note the capitalization
\end{lstlisting}
\end{commandbox}

\subsection{Method 5: Julia Package}

For Julia users, \texttt{prismAId} is available as a package published in the Julia General Registry.\note{The Julia package works on Linux and Windows AMD64, and macOS Arm64.}

\begin{commandbox}[Command: Installing the Julia Package]
\begin{lstlisting}[language=Julia]
using Pkg
Pkg.add("PrismAId")
\end{lstlisting}
\end{commandbox}

\begin{commandbox}[Code: Using prismAId in Julia]
\begin{lstlisting}[language=Julia]
using PrismAId

# Download papers from Zotero
PrismAId.download_zotero_pdfs("username", "api_key", "collection_name", "./papers")

# Download from URL list
PrismAId.download_url_list("urls.txt")

# Convert files to text
PrismAId.convert("./papers", "pdf,docx,html")

# Run a systematic review
toml_config = read("project.toml", String)
PrismAId.run_review(toml_config)
\end{lstlisting}
\end{commandbox}

\section{Setting Up API Keys}

To use the Review tool of \texttt{prismAId}, you need to set up API keys for at least one of the supported LLM providers.\warning{Keep your API keys secure. Never share them in public repositories or unencrypted communications.}

\subsection{Obtaining API Keys}

\begin{itemize}
    \item \textbf{Anthropic}: Register at \href{https://www.anthropic.com/}{anthropic.com} and generate an API key.
    \item \textbf{Cohere}: Sign up at \href{https://cohere.com/}{cohere.com} and retrieve your API key from the dashboard.
    \item \textbf{DeepSeek}: Create an account at \href{https://platform.deepseek.com/}{platform.deepseek.com} and obtain an API key.
    \item \textbf{GoogleAI}: Create an account at \href{https://aistudio.google.com}{aistudio.google.com} and generate an API key.
    \item \textbf{OpenAI}: Register at \href{https://www.openai.com/}{openai.com} and obtain an API key from your account dashboard.
\end{itemize}

\subsection{Configuring API Keys}

There are two ways to configure your API keys:

\subsubsection{Environment Variables}

Set environment variables for your API keys:

\begin{commandbox}[Command: Setting API Key Environment Variables]
\begin{lstlisting}[language=Bash]
# For OpenAI
export OPENAI_API_KEY="your-openai-api-key"

# For GoogleAI
export GOOGLEAI_API_KEY="your-googleai-api-key"

# For Cohere
export COHERE_API_KEY="your-cohere-api-key"

# For Anthropic
export ANTHROPIC_API_KEY="your-anthropic-api-key"

# For DeepSeek
export DEEPSEEK_API_KEY="your-deepseek-api-key"
\end{lstlisting}
\end{commandbox}

\subsubsection{Configuration File}

Add your API keys directly in the TOML configuration file:\warning{Do not share or publish configuration files containing API keys. Always remove them before sharing.}

\begin{configbox}[Configuration: API Keys in TOML]
\begin{lstlisting}[language=TOML]
[project.llm.1]
provider = "OpenAI"
api_key = "your-openai-api-key"
model = "gpt-4o-mini"
temperature = 0.01
tpm_limit = 0
rpm_limit = 0
\end{lstlisting}
\end{configbox}

\tip{If both environment variables and configuration file entries are present, the configuration file values take priority.}

\section{Verifying the Installation}

Check if \texttt{prismAId} is correctly installed:\note{If you encounter issues, refer to the troubleshooting section or seek help from the community through GitHub issues or the Matrix Support Room.}

\begin{commandbox}[Command: Verifying Installation]
\begin{lstlisting}[language=Bash]
# For binaries
./prismaid --help

# For Python
python -c "import prismaid; print(prismaid.__version__)"

# For R
R -e "library(prismaid); cat(prismaid_version())"

# For Julia
julia -e "using PrismAId; println(PrismAId.version())"
\end{lstlisting}
\end{commandbox}

\section{Use in Jupyter Notebooks}

When using prismAId (versions ≤ 0.6.6) in Jupyter notebooks with Python, special handling may be required for interactive prompts:

\begin{commandbox}[Code: Using prismAId v < 0.6.6 in Jupyter Notebooks]
\begin{lstlisting}[language=Python]
import pty
import os
import time
import select

def run_review_with_auto_input(input_str):
    master, slave = pty.openpty()  # Create a pseudo-terminal

    pid = os.fork()
    if pid == 0:  # Child process
        os.dup2(slave, 0)  # Redirect stdin
        os.dup2(slave, 1)  # Redirect stdout
        os.dup2(slave, 2)  # Redirect stderr
        os.close(master)
        import prismaid
        prismaid.RunReviewPython(input_str.encode("utf-8"))
        os._exit(0)

    else:  # Parent process
        os.close(slave)
        try:
            while True:
                rlist, _, _ = select.select([master], [], [], 5)
                if master in rlist:
                    output = os.read(master, 1024).decode("utf-8", errors="ignore")
                    if not output:
                        break  # Process finished

                    print(output, end="")

                    if "Do you want to continue?" in output:
                        print("\n[SENDING INPUT: y]")
                        os.write(master, b"y\n")
                        time.sleep(1)
        finally:
            os.close(master)
            os.waitpid(pid, 0)  # Ensure the child process is cleaned up

# Load your review (TOML) configuration
with open("config.toml", "r") as file:
    input_str = file.read()

# Run the review function
run_review_with_auto_input(input_str)
\end{lstlisting}
\end{commandbox}

\section{Conclusion}

You have now installed \texttt{prismAId} and completed the first-time setup. The toolkit offers five different installation methods, making it accessible across various platforms and programming environments. Each component (Download, Convert, and Review) can be used independently or as part of an integrated workflow.

In the next chapter, we'll explore how to configure your project in detail to leverage all of \texttt{prismAId}'s capabilities.
