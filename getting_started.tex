\chapter{Installation \& Setup} 
\label{chap:install}

This chapter explains how to install \texttt{prismAId}, covering system requirements, installation steps, and first-time setup.

\section{System Requirements}

Before installing \texttt{prismAId}, ensure your system meets the following requirements:\warning{Ensure that you have administrative privileges on your system before installing \texttt{prismAId}, especially on Windows and macOS.}

\begin{itemize}
    \item \textbf{Operating System}: Windows 10 or later, macOS 11 or later, or a Linux distribution (Ubuntu 20.04+, Fedora, Arch, etc.).
    \item \textbf{Processor}: 64-bit CPU (Intel, AMD, or ARM64).
    \item \textbf{Memory}: At least 4GB RAM (8GB recommended).
    \item \textbf{Storage}: Minimum 500MB of free disk space.
\end{itemize}

\section{Installation Instructions}

\texttt{prismAId} can be installed as a standalone executable for direct use.

\subsection{Installing the CLI Executable}

Download the appropriate binary for your operating system from the page of releases in the official repository.

\subsubsection{Windows}
\reminder{On Windows, you may need to allow execution if prompted by security settings.}
\begin{commandbox}[Command: Running \texttt{prismAId} on Windows]
After downloading, navigate to the folder and run:
\begin{lstlisting}[language=Bash]
# Run the program
./prismAId_windows_amd64.exe -project your_project.toml
\end{lstlisting}
\end{commandbox}


\subsubsection{macOS}
\warning{You may need to approve the application in \textbf{System Preferences} under \textbf{Security \& Privacy} before running it.}
\begin{commandbox}[Command: Running \texttt{prismAId} on macOS]
After downloading, give execution permissions and run:
\begin{lstlisting}[language=Bash]
chmod +x prismAId_mac_amd64
./prismAId_mac_amd64 -project your_project.toml
\end{lstlisting}
\end{commandbox}

\subsubsection{Linux}
\tip{If you receive a "Permission denied" error, try running \texttt{chmod +x} again or executing with \texttt{sudo}.}
\begin{commandbox}[Command: Running \texttt{prismAId} on Linux]
\begin{lstlisting}[language=Bash]
chmod +x prismAId_linux_amd64
./prismAId_linux_amd64 -project your_project.toml
\end{lstlisting}
\end{commandbox}


\section{First-Time Setup}

After installation, create a new project configuration.

\begin{commandbox}[Command: Creating a New Project]
Run the following command:
\begin{lstlisting}[language=Bash]
./prismAId_windows_amd64.exe -init
\end{lstlisting}
\end{commandbox}

This prompts the user multiple questions and generates a configuration file.\reminder{Always keep a backup of your configuration file before making major modifications.}

\begin{configbox}[Configuration: Example of Initial Sections of a Configuration]
To configure \texttt{prismAId}, edit the generated \texttt{your\_project.toml} file:
\begin{lstlisting}[language=TOML]
[project]
name = "Use of LLM for systematic review"
author = "John Doe"
version = "1.0"

[project.configuration]
input_directory = "/path/to/txt/files"
input_conversion = ""
results_file_name = "/path/to/save/results"
output_format = "json"
log_level = "low"
duplication = "no"
cot_justification = "no"
summary = "no"
\end{lstlisting}
\end{configbox}

\section{Verifying the Installation}

Check if \texttt{prismAId} is correctly installed:\note{If you encounter issues, refer to the troubleshooting section or seek help from the community.}

\begin{commandbox}[Command: Verifying Installation]
\begin{lstlisting}[language=Bash]
./prismAId_windows_amd64.exe --version
\end{lstlisting}
\end{commandbox}

\section{Conclusion}

You have now installed \texttt{prismAId} and completed the first-time setup. The next chapter will guide you through configuring your project.


\chapter{Configuring prismAId} 
\label{chap:config}

\texttt{prismAId} requires a configuration file to define project settings, data sources, and review parameters. This chapter explains how to initialize, modify, and understand the configuration file.

\section{Initializing a Configuration File}

Users can create a configuration file in two ways:

\subsection{Terminal-Based Initialization}

From the command line, run:

\begin{commandbox}[Command: Initialize a new configuration file]
\begin{lstlisting}[language=Bash]
./prismAId_windows_amd64.exe -init
\end{lstlisting}
\end{commandbox}
\reminder{A configuration template is also provided in the repository for reference.}
This interactively generates a `.toml` configuration file based on user preferences.

\subsection{Web-Based Initialization}

\texttt{prismAId} provides a web-based interface to generate configuration files. Navigate to the a browser-based interface at \href{https://open-and-sustainable.github.io/prismaid/review-configurator.html}{open-and-sustainable.github.io/prismaid/review-configurator.html}, fill in the details interactively, and download or copy the configuration file generated.

\section{Understanding the Configuration File}

The configuration file is structured into sections, each controlling different aspects of \texttt{prismAId}. Below is a breakdown of its components.

\subsection{Project Information}

\begin{configbox}[Configuration: Project Metadata]
\begin{lstlisting}[language=TOML]
[project]
name = "Use of LLM for systematic review"
author = "John Doe"
version = "1.0"
\end{lstlisting}
\end{configbox}

\textbf{Entries:}
\begin{itemize}
    \item \texttt{name} – Title of the project.
    \item \texttt{author} – Name of the person managing the review.
    \item \texttt{version} – Configuration file version.
\end{itemize}

\subsection{Project Configuration}

\begin{configbox}[Configuration: General Settings]
\begin{lstlisting}[language=TOML]
[project.configuration]
input_directory = "/path/to/txt/files"
input_conversion = "pdf,docx"
results_file_name = "/path/to/save/results"
output_format = "json"
log_level = "low"
duplication = "no"
cot_justification = "no"
summary = "no"
\end{lstlisting}
\end{configbox}

\textbf{Entries:}
\begin{itemize}
    \item \texttt{input\_directory} – Path to manuscript files for review.
    \item \texttt{input\_conversion} – File types to be converted (`pdf`, `docx`, etc.).
    \item \texttt{results\_file\_name} – Path for storing output files.
    \item \texttt{output\_format} – Format for results (`csv` or `json`).
    \item \texttt{log\_level} – Logging level (`low`, `medium`, `high`).
    \item \texttt{duplication} – Whether to duplicate manuscripts for debugging (`yes` or `no`).
    \item \texttt{cot\_justification} – Whether to store model justification (`yes` or `no`).
    \item \texttt{summary} – Whether to generate summaries (`yes` or `no`).
\end{itemize}

\subsection{Zotero Integration (Optional)}

\begin{configbox}[Configuration: Zotero Settings]
\begin{lstlisting}[language=TOML]
[project.zotero]
user = ""
api_key = ""
group = ""
\end{lstlisting}
\end{configbox}

\textbf{Entries:}
\begin{itemize}
    \item \texttt{user} – Zotero user ID.
    \item \texttt{api\_key} – API key for accessing Zotero.
    \item \texttt{group} – Collection or group name in Zotero.
\end{itemize}

\subsection{LLM Model Configuration}

\texttt{prismAId} supports multiple LLM providers.

\begin{configbox}[Configuration: LLM Settings]
\begin{lstlisting}[language=TOML]
[project.llm.1]
provider = "OpenAI"
api_key = ""
model = "gpt-4o-mini"
temperature = 0.01
tpm_limit = 0
rpm_limit = 0
\end{lstlisting}
\end{configbox}

\textbf{Entries:}
\begin{itemize}
    \item \texttt{provider} – LLM provider (`OpenAI`, `GoogleAI`, `Cohere`, or `Anthropic`).
    \item \texttt{api\_key} – API key for authentication.
    \item \texttt{model} – Model name (varies by provider).
    \item \texttt{temperature} – Controls randomness (`0-1`, or `0-2` for GoogleAI).
    \item \texttt{tpm\_limit} – Tokens per minute limit (`0` for no limit).
    \item \texttt{rpm\_limit} – Requests per minute limit (`0` for no limit).
\end{itemize}

\subsection{Prompt Configuration}

\begin{configbox}[Configuration: Prompt Structure]
\begin{lstlisting}[language=TOML]
[prompt]
persona = "You are an experienced scientist..."
task = "You are asked to map concepts in a paper."
expected_result = "Output a JSON object with..."
definitions = "'Interest rate' is defined as..."
example = ""
failsafe = "If concepts are unclear, return ''"
\end{lstlisting}
\end{configbox}

\textbf{Entries:}
\begin{itemize}
    \item \texttt{persona} – Role the model should assume.
    \item \texttt{task} – Description of the model’s job.
    \item \texttt{expected\_result} – Defines expected output structure.
    \item \texttt{definitions} – Key concept definitions.
    \item \texttt{example} – Example cases (optional).
    \item \texttt{failsafe} – How the model should handle uncertainty.
\end{itemize}

\subsection{Review Structure}

\begin{configbox}[Configuration: Review Key-Value Structure]
\begin{lstlisting}[language=TOML]
[review.1]
key = "interest rate"
values = [""]
[review.2]
key = "regression models"
values = ["yes", "no"]
\end{lstlisting}
\end{configbox}

\textbf{Entries:}
\begin{itemize}
    \item \texttt{key} – Topic being analyzed.
    \item \texttt{values} – Possible values for each key.
\end{itemize}

\section{Conclusion}

The configuration file defines every aspect of a \texttt{prismAId} project. Users have no need to define or modify anything else. 

Users can modify settings manually starting from the provided template, or use the web configurator or the terminal-based one for an interactive setup.

