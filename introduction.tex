\chapter{Introduction to prismAId} 
\label{chap:intro}

Systematic reviews are at the core of evidence-based research. They help synthesize vast amounts of literature, ensuring that decisions in science, medicine, policy, and other fields are grounded in the best available evidence. However, conducting systematic reviews is a time-intensive and often overwhelming process. Screening literature, managing citations, extracting data, and ensuring methodological rigor all demand meticulous attention to detail.

\texttt{prismAId} was created to address these challenges. It is an open-source tool designed to assist researchers in managing systematic reviews more efficiently.\tip{Systematic reviews require structured workflows. \texttt{prismAId} provides tools to help enforce best practices.} By streamlining key aspects of the review process, \texttt{prismAId} helps users maintain high standards of rigor and reproducibility while reducing the manual workload.

The tool is aimed at a diverse range of users, from researchers conducting large-scale systematic reviews to students working on literature-based projects. It provides structured workflows that align with best practices in systematic reviewing, ensuring that every step—from defining inclusion criteria to extracting and analyzing data—is traceable and transparent.

\bigskip

One of the key motivations behind \texttt{prismAId} is its commitment to Open Science. The tool is fully open source, meaning its development is transparent, and researchers can inspect, modify, and contribute to its functionality. This openness ensures that systematic reviews conducted with \texttt{prismAId} can be fully reproducible and that researchers can collaborate effectively without relying on proprietary or closed software ecosystems.\note{Open-source software allows full transparency. You can verify how \texttt{prismAId} processes data at any time.}

Moreover, the open-source nature of \texttt{prismAId} allows for continuous improvement through community-driven development. Users can adapt the tool to fit their needs, extend its capabilities, and integrate it with other research software.\warning{As an open-source tool, \texttt{prismAId} does not include proprietary support. Users are encouraged to engage with the community for troubleshooting and development.} This flexibility is particularly beneficial in a rapidly evolving scientific landscape where reproducibility and adaptability are crucial.

Beyond technical efficiency, \texttt{prismAId} embodies key guiding principles that impact its users:

\begin{itemize}
    \item \textbf{Transparency}: The logic behind how studies are managed and processed is open for review. There are no hidden decision-making mechanisms.
    \item \textbf{Reproducibility}: Every action taken within \texttt{prismAId} can be logged and traced, allowing others to replicate results.
    \item \textbf{Flexibility}: Researchers can configure \texttt{prismAId} to meet the specific requirements of their systematic review protocols.
    \item \textbf{No Vendor Lock-in}: Users are not dependent on a single commercial provider, ensuring that research workflows remain independent.
    \item \textbf{Community and Collaboration}:\reminder{Collaboration is a core feature. Consider contributing to \texttt{prismAId} if you have ideas for improvement!} As an open-source tool, \texttt{prismAId} fosters collaboration among researchers, developers, and systematic review specialists.
\end{itemize}


By embracing these principles, \texttt{prismAId} is more than just a tool—it is part of a broader effort to improve the accessibility, efficiency, and reliability of systematic reviews. Whether used by a lone researcher or a large team, it provides the structure and support needed to navigate complex literature and synthesize high-quality evidence.

As we move through this manual,\note{This manual follows a structured approach. Refer to the \textbf{How Should You Use This Manual?} section in the Foreword for guidance on navigating the content.} you will learn how to install, configure, and effectively use \texttt{prismAId} to conduct systematic reviews. From getting started with basic setup to leveraging advanced features for more complex reviews, this guide will provide all the necessary information to integrate \texttt{prismAId} into your research workflow.


